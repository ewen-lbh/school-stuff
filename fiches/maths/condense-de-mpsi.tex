% Basic stuff
\documentclass{article}
\usepackage[utf8]{inputenc}
\usepackage[a4paper, total={6.5in, 9.5in}]{geometry}
\usepackage[bookmarks, hidelinks, unicode]{hyperref}
\usepackage[]{amsmath,amssymb}
\usepackage{tikz}
\usepackage{lmodern}
\usepackage{soul}

% Packages configuration
\usetikzlibrary{shapes.arrows}
\renewcommand{\arraystretch}{1.4}

% Shortcut commands
\newcommand{\im}{\text{Im}\,}
\newcommand{\re}{\text{Re}\,}
\newcommand{\img}{\text{Img}\,}
\newcommand{\R}{{\mathbb R}}
%\newcommand{\C}{{\mathbb C}}
\newcommand{\N}{{\mathbb N}}
\newcommand{\Z}{{\mathbb Z}}
\newcommand{\Q}{{\mathbb Q}}
\newcommand{\cotan}{\operatorname{cotan}}
\newcommand{\conj}[1]{\overline{#1}}
\newcommand{\Aff}{\text{Aff}}
\newcommand{\twoRows}[1]{\multirow{2}{*}{#1}}
\newcommand{\threeRows}[1]{\multirow{3}{*}{#1}}
\newcommand{\twoCols}[1]{\multicolumn{2}{c|}{#1}}
\newcommand{\threeCols}[1]{\multicolumn{3}{|c|}{#1}}
\newcommand{\twoColsNB}[1]{\multicolumn{2}{c}{#1}}
\newcommand{\goesto}[2]{\xrightarrow[#1\:\to\:#2]{}}
\newcommand{\liminfty}{\lim_{x\to+\infty}}
\newcommand{\limminfty}{\lim_{x\to-\infty}}
\newcommand{\limzero}{\lim_{x\to0}}
\newcommand{ \const}{\text{cste}}
\newcommand{\et}{\:\text{et}\:}
\newcommand{\ou}{\:\text{ou}\:}
\newcommand{\placeholder}{\diamond}
\newcommand{\mediateur}{\:\text{med}\:}
\newcommand{\milieu}{\:\text{mil}\:}
\newcommand{\vect}[1]{\overrightarrow{#1}}
\newcommand{\point}[2]{(#1;\;#2)}
\newcommand{\spacepoint}[3]{\begin{pmatrix}#1\\#2\\#3\end{pmatrix}}
\newcommand{\sh}{\operatorname{sh}}
\newcommand{\ch}{\operatorname{ch}}
\renewcommand{\th}{\operatorname{th}}
\newcommand{\id}{\operatorname{id}}
\renewcommand{\cong}{\equiv}

% Document
\begin{document}

\begin{titlepage}
	\begin{center}
		\textit{\today}
		\vfill
		\textbf{\LARGE{Condensé de la MPSI}\\\Large{Mathématiques}}\\
		\vfill
		\large{Ewen Le Bihan\\MPSI -- Daudet}
	\end{center}
\end{titlepage}

\newpage
\tableofcontents
%\pagestyle{empty}
\newpage

\section{Processus de démonstration}
\subsection{Processus élémentaires}
\subsubsection{Quantification universelle $\forall$}
\label{quantification_universelle}
Soit $a \in E$

\subsubsection{Quantification existentielle $\exists$}
\label{quantification_existentielle}
Posons $a = \ldots \in E$

\subsubsection{Quantification existentielle unique $\exists!$}
\paragraph{Existence}
\emph{cf. \ref{quantification_existentielle}} 

\paragraph{Unicité}
Posons $b \in E$.
\emph{Démonstration de $b = a$} 

\subsubsection{Implication $P \implies Q$}
\label{implication}
Supposons $P(a)$. Montrons $Q(a)$

\subsubsection{Équivalence $P \iff Q$}
Procédons par double implication.

$\implies$: \emph{Démonstration de $P \implies Q$} 

$\impliedby$: \emph{Démonstration de $P \impliedby Q$} 

\subsubsection{Inclusion $E \subset F$}
\emph{Démontrer $\forall x \in \mathbb{E}, x \in E \implies x \in F$}.

\subsubsection{Égalité ensembliste}
Procédons par double inclusion.

$\subset$: Démonstration de $E \subset F$

$\supset$: Démonstration de $E \supset F$


\subsection{Processus de démonstration}
On commence chaque démonstration utilisant un de ces processus par << Procédons par \emph{nom du processus}  >> 

\subsubsection{Récurrence}
\emph{Pour montrer une propriété vraie dans $E \subseteq \N$} 

\paragraph{Initialisation}
\emph{Démontrer la propriété au premier rang} 

\paragraph{Hérédité}
Démontrer $\forall n \in E, P(n) \implies P(n+1)$

\paragraph{Conclusion}
La propriété étant initialisée et héréditaire, elle est vraie pour tout $n \in E$.

\subsubsection{Contraposée}
\emph{Pour montrer $P \implies Q$ quand l'implication directe est trop compliquée}

\emph{Démontrer $\lnot Q \implies \lnot P$} 

\subsubsection{l'Absurde}

\emph{Pour montrer $P$} 

Supposons $\lnot P$

$\vdots$

On obtient une contradiction.

On a donc $P$

\subsubsection{Disjonction des cas}

\paragraph{1er cas: \ldots} \ldots

\paragraph{2ème cas: \ldots} \ldots

$\vdots$

\paragraph{$n$-ième cas: \ldots} \ldots

\paragraph{Conclusion}
\ldots

\subsubsection{Analyse-Synthèse}
\emph{Pour trouver les solutions d'une équation, inéquation, \ldots}

\paragraph{Analyse}
Soit $a \in E$. Supposons $P(a)$.

\emph{Réduire le nombre de candidats possibles pour $a$} 

\paragraph{Synthèse}
Testons nos candidats

\paragraph{Conclusion}
Les solutions sont \ldots

\newpage
\section{Dérivation}
\emph{Attention aux hypothèses!} 
\subsection{Nombre dérivé en un point}
\[
	f'(a) = \lim_{x \to a} \frac{f(x)-f(a)}{x-a}
\] 

\subsection{Dérivée de $f$}
\[
	f' = \begin{cases}
		I\to \R \\
		a \mapsto f'(a)
	\end{cases}
\] 

\subsection{Dérivée usuelles}
\begin{itemize}
	\item $\forall n \in \N,\quad  (\text{id}^n)' = n \text{id}^{n-1}$
	\item $\forall n \in \N,\quad \sqrt[n]{\;}' = \frac{1}{n\sqrt[n]{\;} }  $ 
	\item $\ln' = \frac{1}{\text{id}}$
	\item $\exp'=\exp$
	\item $(a^\text{id})' = x\mapsto \ln(a)a^x$
	\item $ \sin' = \cos $ 
	\item $\cos' = -\sin$
	\item $\tan' = \frac{1}{\cos^2} = 1 + \tan^2$
	\item $ \sh' = \ch $
	\item $ \ch' = \sh$
	\item $\th' = \frac{1}{\ch^2} = 1 + \th^2$
	\item $\text{acos}' = \frac{-1}{\sqrt{1-\id^2} }$
	\item $\text{asin}' = \frac{1}{\sqrt{1-\id^2} }$
	\item $\text{atan}' = \frac{1}{1 + \id^2}$
\end{itemize}

\subsection{Dérivées de composées}
\begin{itemize}
	\item $\forall (\lambda, \mu) \in \R^2,\quad (\lambda u + \mu v)' = \lambda u' + \mu v'$
	\item $(uv)' = u'v+v'u$
	\item $(\frac{1}{v})' = \frac{-v'}{v^2}$
	\item $(\frac{u}{v})' = \frac{u'v-v'u}{v^2}$
	\item $(u \circ v)' = v' \cdot  (u' \circ v)$
	\item $(u^{-1})' = \frac{1}{u' \circ u^{-1}}$
\end{itemize}

\section{Trigonométrie}
\subsection{Cercle trigonométrique ou unité $\mathcal C$}

Cercle de centre $\point{0}{0}$ et de rayon $1$.

\[
	\mathcal C = \{ \point{x}{y} \in \R^2,\: x^2 + y^2 = 1 \} = \{ \point{\cos x}{\sin x},\: x \in \R \} 
\] 

\subsection{Congruence $\cdot \cong \cdot [\cdot]$}

\[
	a \cong b\ [t] \stackrel{\text{def}}{\iff} \exists k\in \Z,\ a = b + kt

\] 
\subsubsection{Propriétés}

\begin{itemize}
	\item $\forall a, b, c, d\in \R,\ \begin{cases}
			a \cong b\ [t]\\
			c \cong d\ [t]
		\end{cases} \implies a + c \cong c + d\ [t]$
	\item $\forall a, b, \lambda \in \R,\ a\cong b\ [t] \implies \lambda a \cong \lambda b\ [\lambda t] \ \et \begin{cases}
			\lambda a \cong \lambda b\ [t] \\
			\lambda \in \Z
	\end{cases}$
\item $ \cdot \cong \cdot \ [ \cdot ]$ est \st{un RAT} une relation d'équivalence
\end{itemize}



\end{document}
