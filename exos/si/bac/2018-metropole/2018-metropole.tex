\documentclass{article}
\usepackage[utf8]{inputenc}
\usepackage[a4paper, total={6.5in, 9.5in}]{geometry}
\usepackage{float}
\usepackage{amsmath}
\usepackage{xcolor}
\usepackage{amssymb}
\usepackage{mathtools}
\usepackage{tensor}
\newcommand{\torseur}[7]{
\tensor[_{#1}]{\left\{ \begin{array}{cc}
    #2 & #5 \
    #3 & #6 \
    #4 & #7
\end{array} \right\}}{_{(\vec{x};\vec{y};\vec{z})}}
}
\usepackage{siunitx}
\sisetup{output-decimal-marker={,},group-minimum-digits=4,abbreviations}
\sisetup{inter-unit-product=\ensuremath{{}\cdot{}}}
\newcommand{\deftable}[2]{%
%\hline
\textbf{B.A.M.E}
\begin{table}[h]
    \centering
    \begin{tabular}{llp{130mm}}%
        %& unité/type & Explication \ \hline
        #1
    \end{tabular}
    \label{tab:#2_units}
\end{table}%
}
\newcommand{\deftablevar}[3]{%
    $#1$ & $\si{#2}$ & #3 \
}
\newcommand{\deftableobj}[3]{%
    $#1$ & \textit{#2} & #3 \
}
\newcommand{\bame}[1]{%
%\hline
\begin{table}[h]
    \centering
    \begin{tabular}{llllp{130mm}}%
        Nom & Vecteur & Direction & Sens & Norme \hline
        #1
    \end{tabular}
\end{table}%
}
\newcommand{\vect}[1]{\overrightarrow{#1}}

\title{2018 Métropole}
\author{Ewen Le Bihan}
\date{2020-06-12}

\begin{document}

\maketitle
\setcounter{section}{13}
\section{}
Par lecture graphique, 
\[
	E_{\text{revalorisée}} \approx \SI{110}{\mega\joule}
\] 
\section{}

\begin{align*}
	E_{\text{restituée}} &= E_{\text{revalorisée}} \cdot \eta_\text{mv}\color{green}^2 \\
			     &\approx 110 \cdot 0.84 \\
			     &\approx \SI{92.4}{\mega\joule}
\end{align*}

\section{}
Par lecture graphique, l'écart maximal entre $E_\text{restituée}$ et $E_\text{simulée}$ est à $t = \SI{250}{\second}$.

\paragraph{}
À $t = \SI{250}{\second}$:

\begin{align*}
	\text{ER} &= \frac{E_\text{simulée} - E_\text{mesurée}}{E_\text{simulée}} \\
	&= \frac{25 - 17.5}{25} \\
	&= 0.3\quad\text{soit}\quad 30\% \\
	&\implies \color{green}{\text{le modèle n'est pas valide}}
\end{align*}

\section{}
Il réduire le paramètre $J_{\text{SSI}}$: la vitesse ralentit trop lentement et augmente trop, en baissant l'intertie, on arrivera à un stockage d'énergie moins élevé, car le système aura enmagasiné moins d'énergie, et aura pour conséquence une décélération plus forte, puisque moins d'énergie à évacuer.

\textcolor{green}{Il faut augmenter le couple de frottement}

\section{}
Par lecture graphique,
\[
	\sigma_\text{maxi} = \SI{462}{\mega\pascal}
\] 

\emph{Correction à partir de là} 

\section{}

L'énergie cinétique stockée est de $\SI{38.6}{\mega\joule}$ d'après la figure 8. La variation d'énergie cinétique maximale est $\Delta E_c = 38.6 - 6.7 = \SI{31.9}{\mega\joule}$ 

La valeur $J_{\text{SSI}}$ paramétrée dans le modèle est le moment d'inertie équivalent ramené sur l'axe moteur: $J_\text{SSI} = \SI{376}{\kilo\gram\per\meter\squared}$.

Or $E_{c\text{max}} = \frac{1}{2}J_{\text{SSI}}\omega^2_\text{max}$ 

La vitesse de rotation maximale du volant d'inertie vaut: $\omega_\text{max} = \sqrt{\frac{2E_{c\text{max}}}{J_\text{SSI}}} = \SI{453}{\radian\per\second}$

\section{}
\begin{align*}
	\text{CS} &= \frac{R_e}{\sigma_\text{maxi}} \\
	&= \frac{551.5}{462} \\
	&= 1.2 \\
	&< 2 \\
	&\implies \text{Le volant n'est pas en mesure de supporter cette survitesse}
\end{align*}

L'énergie excédentaire sera donc dissipée par les rhéostats

\section{}

\section{}
La loi d'entrée-sortie du capteur donne l'équation de droite ci-dessous

\[
	U_e = 0.25 \cdot T_e + 2.5
\] 

Pour $T_e = 4\degree C$

\[
	U_e = \SI{3.5}{\volt}
\] 

\section{}
\begin{align*}
	q &= \frac{Ue_\text{max}-Ue_\text{min}}{2^{n}} \\
	  &= \frac{10-0}{2^{n}} \\
	  &= \SI{0.039}{\volt}
\end{align*}

\emph{il manque un truc} 

\section{}

\ldots \\
Si $T_e \leq 89(10)$ \\
\ldots \\
Alors $Ch \leftarrow 0$ \\
\ldots

\section{}
Amplification statique: $K = \frac{\Delta S}{\Delta E} = \frac{12}{750} = \SI{0.016}{\celsius\per\volt}$

$\tau = 70 - 10 = \SI{60}{\second}$

\section{}

\begin{itemize}
	\item La simulation n°1 correspond au fonctionnement attendu
	\item $T = \SI{50}{\second}$ et le rapport cyclique $\alpha=\frac{t_\text{ON}}{T} = \frac{33}{50} = 0.66$ 
	\item gain énergétique réalisable: $(1-0.66) \cdot 100 = 34\%$

\end{itemize}

\section{}
\emph{Rédiger une phrase\ldots} 
\begin{itemize}
	\item dénivelés importants dans zones aériennes (adhérance des roues)
	\item intervalles prohibés
	\item incidents mineurs
	\item Mettre en place 2e SSI ou rempl volant par plsu important
	\item Limiter zones ext pour limiter chauffe des voies
\end{itemize}

\end{document}
