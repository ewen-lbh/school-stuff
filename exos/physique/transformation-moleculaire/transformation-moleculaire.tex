\documentclass{article}
\usepackage[utf8]{inputenc}
\usepackage[a4paper, total={6.5in, 9.5in}]{geometry}
\usepackage{hyperref}
\usepackage{chemfig}
\usepackage{amsmath, amssymb, dsfont}
\usepackage{mathtools}
\usepackage{chemfig}
\usepackage{chemformula}
\usepackage{xcolor}
\definecolor{gray07}{gray}{0.7}
\usepackage{float}
\usepackage{enumitem}
\usepackage{cancel}
\usepackage{pgfplots}
\usepgfplotslibrary{units} % to add units easily to axis
\usepgfplotslibrary{fillbetween} % to fill inbetween curves
\usepgfplotslibrary{colormaps} % to create colormaps
\pgfplotsset{width=12.2cm, height=7cm}
\pgfplotsset{compat=newest} %(making it only compatalbe with
%new releases of pgfplots)
\pgfdeclarehorizontalshading{visiblelight}{50bp}{
color(0.00000000000000bp)=(violet);
color(8.33333333333333bp)=(blue);
color(16.66666666666670bp)=(cyan);
color(25.00000000000000bp)=(green);
color(33.33333333333330bp)=(yellow);
color(41.66666666666670bp)=(orange);
color(50.00000000000000bp)=(red)
}%
\usepackage{tikz}
\usetikzlibrary{decorations.pathreplacing}
\usepackage{multirow}
\usepackage{caption}
\usepackage[load-configurations = abbreviations]{siunitx}
\sisetup{inter-unit-product=\ensuremath{{}\cdot{}}, exponent-product=\ensuremath{{}\cdot{}}, group-minimum-digits=3, range-phrase=-, range-units=single}
\usepackage{graphicx}
\graphicspath{ {./} }
\usetikzlibrary{shapes.arrows}
\let\ce\ch
\newcommand{\bit}{\text{bit}}
\newcommand{\equilibrium}[2]{$\ce{#1}\rightleftharpoons\ce{#2}$}
\newcommand{\soom}{\stackrel{10}{=}}
\newcommand{\vect}{\overrightarrow}
\newcommand{\formulaPyramidFactory}[4]{
    \begin{tikzpicture}[scale=1.25]
\draw (0,0) -- (-0.5, -0.75) node  [align=center, fill=white, midway] {\footnotesize \---};
\draw (0,0) -- (0.5, -0.75) node [align=center, fill=white, midway] {\footnotesize \---};
\draw (0.5, -0.75) -- (-0.5, -0.75) node [align=center, fill=white, midway] {\footnotesize #4};

\draw (0,0) node [above, fill=white] {#1};
\draw (-0.5, -0.75) node [left, fill=white] {#2};
\draw (0.5, -0.75) node [right, fill=white] {#3};

%\draw (0, -1) node [below, align=center] {\small (Pyramide de formule)};
\end{tikzpicture}
}
\newcommand{\formulaPyramid}[3]{\formulaPyramidFactory{#1}{#2}{#3}{$\times$}}
\newcommand{\formulaPyramidAdditive}[3]{\formulaPyramidFactory{#1}{#2}{#3}{$+$}}
\newenvironment{definitions}{\begin{description}[leftmargin=!,labelwidth=\widthof{\bfseries Lorem ipsum dolor}]}{\end{description}}
\newcommand{\deftable}[2]{%
%\hline
\begin{table}[h]
    \centering
    \begin{tabular}{llp{100mm}}%
        %& unité/type & Explication \\ \hline
        #1
    \end{tabular}
    \label{tab:#2_units}
\end{table}%
}
\newcommand{\deftablevar}[3]{%
    $#1$ & $\si{#2}$ & #3 \\
}
\newcommand{\deftableobj}[3]{%
    $#1$ & \textit{#2} & #3 \\
}
\newcommand{\chg}[2]{\charge{45:1.5pt=$\scriptstyle#1$}{#2}}

% =======================
% SECTIONS FORMATTING
% =======================
\renewcommand{\thesection}{}
\renewcommand{\thesubsection}{\arabic{subsection}}
\renewcommand{\thesubsubsection}{\alph{subsubsection})}
\makeatletter
\def\@seccntformat#1{\csname #1ignore\expandafter\endcsname\csname the#1\endcsname\quad}
\let\sectionignore\@gobbletwo
\let\latex@numberline\numberline
\def\numberline#1{\if\relax#1\relax\else\latex@numberline{#1}\fi}
\makeatother

\title{Exercices: Transformation moléculaire}
\date{2020-05-18}
\author{Ewen Le Bihan}

\begin{document}

\maketitle

\section{@300 15}

\subsection{} a, b
\subsection{} a
\subsection{} a

\section{@300 16}

\begin{table}[h]
	\centering
	\begin{tabular}{c|c}
		Molécule & Type de transformation \\\hline
		a & élimination \\
		b & addition \\
		c & substitution \\
		d & élimination \\
		e & \\
		f & addition \\
	\end{tabular}
\end{table}

\section{@316 }

\newpage
\section{@320 20}

\begin{enumerate}
	\item \chemfig{[0,.7]
		H_3C-C(=[6]\charge{-135=\|,-45=\|}{O})-\charge{90=\|,-90=\|}{O}-CH_2-CH_3
			+\charge{90=\|,-90=\|,180=\|,45:1.5pt=$\scriptstyle\ominus$}{O}-[0,1.8]H \quad\to\quad
			H_3C-C(-[6]\charge{45:1.5pt=$\scriptstyle\ominus$,180=\|,-90=\|,0=\|}{O})(-[2]O(-[2]O))-\charge{90=\|,-90=\|}{O}-CH_2-CH_3
	}
	\item \chemfig{
			H_3C-C(-[6]\charge{45:1.5pt=$\scriptstyle\ominus$,180=\|,-90=\|,0=\|}{O})(-[2]O(-[2]H))-\charge{90=\|,-90=\|}{O}-CH_2-CH_3 \quad\to\quad
			H_3C-C(=[6]\charge{-135=\|,-45=\|}{O})-\charge{-90=\|,90=\|}{O}-H + \chg{\ominus}{O}-CH_2-CH_3
	}
	\item \chemfig{
		H_3C-C(=[6]\charge{-135=\|,-45=\|}{O})-\charge{-90=\|,90=\|}{O}-H +  
			\charge{45:1.5pt=$\scriptstyle\ominus$,90=\|,-90=\|,180=\|}{O}-CH_2-CH_3 \quad\to\quad
		H_3C-C(=[6]\charge{-135=\|,-45=\|}{O})-\charge{45:1.5pt=$\scriptstyle\ominus$,0=\|,-90=\|,90=\|}{O}+H-\charge{90=\|,-90=\|}{O}
			-CH_2-CH_3
	}
\end{enumerate}

\section{@320 21}
\subsection{}

\begin{align*}
	\chemfig{C_4H_{10}O} \overset{\text{acide}}{\quad\to\quad} \chemfig{C_4H_7Cl} + \chemfig{H_2O} 
\end{align*}

\subsection{}
\subsubsection{}

\begin{description}
	\item[Étape 1] déprotonation
	\item[Étape 2] élimination d'eau
	\item[Étape 3] protonation
\end{description}

\subsubsection{}


\[
	\chemfig{H_3C-C(-[6]\charge{0=\|,180=\|}{O}(-[6]H))(-[2]CH_3)-CH_3 + \chg{\oplus}{H}
	\quad\to\quad H_3C-C(-[6]\charge{45:1.5pt=$\scriptstyle\oplus$,180=\|}{O}(-[6]H)-H)(-[2]CH_3)-CH_3}
\]

\subsubsection{}
\paragraph{Étape 1}
On a $A = \chemfig{H_3C-C(-[2]CH_3)-CH_3}$
\[
	\chemfig{H_3C-C(-[6]\charge{45:1.5pt=$\scriptstyle\oplus$,180=\|}{O}(-[6]H)-H)(-[2]CH_3)-CH_3 
	\quad\to\quad H_3C-C(-[2]CH_3)-CH_3 + H-[1]\charge{45=\|,135=\|}{O}-[7]H}
\]

\[
	\chemfig{H_3C-C(-[2]CH_3)-CH_3} \quad\to\quad 
	\chemfig{C(-[3]H)(-[5]H)=C(-[1]CH_3)(-[7]CH_3)} + \chemfig{\chg{\oplus}{H}}
\] 

\end{document}
