\documentclass{article}
\usepackage[utf8]{inputenc}
\usepackage[a4paper, total={6.5in, 9.5in}]{geometry}
\usepackage[bookmarks,hidelinks]{hyperref}
\usepackage{amsmath, amssymb, dsfont}
\usepackage{mathtools}
\usepackage{chemfig}
\usepackage{chemformula}
\usepackage{xcolor}
\definecolor{gray07}{gray}{0.7}
\usepackage{float}
\usepackage{enumitem}
\usepackage{cancel}
\usepackage{pgfplots}
\usepgfplotslibrary{units} % to add units easily to axis
\usepgfplotslibrary{fillbetween} % to fill inbetween curves
\usepgfplotslibrary{colormaps} % to create colormaps
\pgfplotsset{width=12.2cm, height=7cm}
\pgfplotsset{compat=newest} %(making it only compatalbe with
%new releases of pgfplots)
\pgfdeclarehorizontalshading{visiblelight}{50bp}{
color(0.00000000000000bp)=(violet);
color(8.33333333333333bp)=(blue);
color(16.66666666666670bp)=(cyan);
color(25.00000000000000bp)=(green);
color(33.33333333333330bp)=(yellow);
color(41.66666666666670bp)=(orange);
color(50.00000000000000bp)=(red)
}%
\usepackage{tikz}
\usetikzlibrary{decorations.pathreplacing}
\usepackage{multirow}
\usepackage{caption}
\usepackage[load-configurations = abbreviations]{siunitx}
\sisetup{inter-unit-product=\ensuremath{{}\cdot{}}, exponent-product=\ensuremath{{}\cdot{}}, group-minimum-digits=3, range-phrase=-, range-units=single}
\usepackage{graphicx}
\graphicspath{ {./} }
\usetikzlibrary{shapes.arrows}
\let\ce\ch
\newcommand{\bit}{\text{bit}}
\newcommand{\equilibrium}[2]{$\ce{#1}\rightleftharpoons\ce{#2}$}
\newcommand{\soom}{\stackrel{10}{=}}
\newcommand{\vect}{\overrightarrow}
\newcommand{\formulaPyramidFactory}[4]{
    \begin{tikzpicture}[scale=1.25]
\draw (0,0) -- (-0.5, -0.75) node  [align=center, fill=white, midway] {\footnotesize \---};
\draw (0,0) -- (0.5, -0.75) node [align=center, fill=white, midway] {\footnotesize \---};
\draw (0.5, -0.75) -- (-0.5, -0.75) node [align=center, fill=white, midway] {\footnotesize #4};

\draw (0,0) node [above, fill=white] {#1};
\draw (-0.5, -0.75) node [left, fill=white] {#2};
\draw (0.5, -0.75) node [right, fill=white] {#3};

%\draw (0, -1) node [below, align=center] {\small (Pyramide de formule)};
\end{tikzpicture}
}
\newcommand{\formulaPyramid}[3]{\formulaPyramidFactory{#1}{#2}{#3}{$\times$}}
\newcommand{\formulaPyramidAdditive}[3]{\formulaPyramidFactory{#1}{#2}{#3}{$+$}}
\newenvironment{definitions}{\begin{description}[leftmargin=!,labelwidth=\widthof{\bfseries Lorem ipsum dolor}]}{\end{description}}
\newcommand{\deftable}[2]{%
%\hline
\begin{table}[h]
    \centering
    \begin{tabular}{llp{100mm}}%
        %& unité/type & Explication \\ \hline
        #1
    \end{tabular}
    \label{tab:#2_units}
\end{table}%
}
\newcommand{\deftablevar}[3]{%
    $#1$ & $\si{#2}$ & #3 \\
}
\newcommand{\deftableobj}[3]{%
    $#1$ & \textit{#2} & #3 \\
}

\title{Le Kicking Sport}
\date{2020-06-08}
\author{Ewen Le Bihan}

\begin{document}
\begin{abstract}
	Normal, \emph{Correction} 
\end{abstract}

\maketitle

\section{Modélisation du mouvement de la structure}
\subsection{}
\begin{description}
	\item[$\vec F_1$] Réaction du support
	\item[$\vec F_2$] Poids
\end{description}

\begin{description}
	\item[Cas n°1] À l'endroit \emph{En train de se balancer} 
	\item[Cas n°2] À l'envers \emph{Au repos} 
\end{description}

\subsection{}
\texttt{vector from sphere to soil label vec F\_1; vector from sphere to summit vec F\_2}

\emph{Correction à partir de là} 

\[
 \vec v \begin{cases}
		 \text{Direction} & \text{Tangeante à la trajectoire} \\
		 \text{Sens} & \text{Sens du mouvement} \\
		 \text{Norme} & \text{Variable} \\
		 \text{Pt d'application} & \text{Masse }m \\
	 \end{cases}
\] 

\subsection{}
\subsubsection{}

\begin{align*}
	a &= x \quad\text{Car $x$ alterne entre valeurs positives et négatives} \\
	\implies b &= y \quad\text{(par élimination)}\\
	T_1&= \SI{5}{s} \\
	T_2&= \SI{2.5}{s} \\
\end{align*}

\subsubsection{}

\begin{align*}
	T &= 2\pi\sqrt{\frac{l}{g}}  \\
	&= 2\pi\sqrt{\frac{4.15}{9.81}}  \\
	&= \SI{5.37}{s} \\
\end{align*}

$T_1$ représente la période des oscillations.

\subsubsection{}
\emph{Le prof avait la flemme de rédiger, en gros c'est la relation entre $x$ et $y$} 

\section{Étude énergétique du mouvement du système}

\subsection{}
\begin{description}
	\item[Pointillés] $E_p$ 
	\item[Continu] $E_c$
\end{description}

\subsection{}
\[
	E_m &= E_c + E_p \\
\] 

\texttt{graph \{ line f(t) = 900 \}}

\subsection{}

\begin{align*}
	h &= l - l\cos \theta \\
	  &= l(1 - \cos \theta) \\
\end{align*}

\begin{align*}
	E_m &=  E_c + E_{pp} \\
	&= \frac{1}{2}mv^2+mgh \\
	&= \frac{1}{2}m(l\omega)^2 + mgh \\
	&= \frac{1}{2}m\left( l\frac{d \theta}{dt} \right)^2 + mgl(1-\cos \theta)  \\
	&= \frac{1}{2}ml\left( \frac{d \theta}{dt} \right)^2 + mgl(1-\cos \theta)  \\
	&\text{On dérive\ldots}\\
	%&= \frac{1}{2}ml^2 2\theta + mgl\sin \theta\\ j'ai pas pu suivre
\end{align*}

\end{document}
