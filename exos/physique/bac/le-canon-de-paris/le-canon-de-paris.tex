\documentclass{article}
\usepackage[utf8]{inputenc}
\usepackage[a4paper, total={6.5in, 9.5in}]{geometry}
\usepackage[bookmarks,hidelinks]{hyperref}
\usepackage{amsmath, amssymb, dsfont}
\usepackage{mathtools}
\usepackage{chemfig}
\usepackage{chemformula}
\usepackage{xcolor}
\definecolor{gray07}{gray}{0.7}
\usepackage{float}
\usepackage{enumitem}
\usepackage{cancel}
\usepackage{pgfplots}
\usepgfplotslibrary{units} % to add units easily to axis
\usepgfplotslibrary{fillbetween} % to fill inbetween curves
\usepgfplotslibrary{colormaps} % to create colormaps
\pgfplotsset{width=12.2cm, height=7cm}
\pgfplotsset{compat=newest} %(making it only compatalbe with
%new releases of pgfplots)
\pgfdeclarehorizontalshading{visiblelight}{50bp}{
color(0.00000000000000bp)=(violet);
color(8.33333333333333bp)=(blue);
color(16.66666666666670bp)=(cyan);
color(25.00000000000000bp)=(green);
color(33.33333333333330bp)=(yellow);
color(41.66666666666670bp)=(orange);
color(50.00000000000000bp)=(red)
}%
\usepackage{tikz}
\usetikzlibrary{decorations.pathreplacing}
\usepackage{multirow}
\usepackage{caption}
\usepackage[load-configurations = abbreviations]{siunitx}
\sisetup{inter-unit-product=\ensuremath{{}\cdot{}}, exponent-product=\ensuremath{{}\cdot{}}, group-minimum-digits=3, range-phrase=-, range-units=single}
\usepackage{graphicx}
\graphicspath{ {./} }
\usetikzlibrary{shapes.arrows}
\let\ce\ch
\newcommand{\bit}{\text{bit}}
\newcommand{\equilibrium}[2]{$\ce{#1}\rightleftharpoons\ce{#2}$}
\newcommand{\soom}{\stackrel{10}{=}}
\newcommand{\vect}{\overrightarrow}
\newcommand{\formulaPyramidFactory}[4]{
    \begin{tikzpicture}[scale=1.25]
\draw (0,0) -- (-0.5, -0.75) node  [align=center, fill=white, midway] {\footnotesize \---};
\draw (0,0) -- (0.5, -0.75) node [align=center, fill=white, midway] {\footnotesize \---};
\draw (0.5, -0.75) -- (-0.5, -0.75) node [align=center, fill=white, midway] {\footnotesize #4};

\draw (0,0) node [above, fill=white] {#1};
\draw (-0.5, -0.75) node [left, fill=white] {#2};
\draw (0.5, -0.75) node [right, fill=white] {#3};

%\draw (0, -1) node [below, align=center] {\small (Pyramide de formule)};
\end{tikzpicture}
}
\newcommand{\formulaPyramid}[3]{\formulaPyramidFactory{#1}{#2}{#3}{$\times$}}
\newcommand{\formulaPyramidAdditive}[3]{\formulaPyramidFactory{#1}{#2}{#3}{$+$}}
\newenvironment{definitions}{\begin{description}[leftmargin=!,labelwidth=\widthof{\bfseries Lorem ipsum dolor}]}{\end{description}}
\newcommand{\deftable}[2]{%
%\hline
\begin{table}[h]
    \centering
    \begin{tabular}{llp{100mm}}%
        %& unité/type & Explication \\ \hline
        #1
    \end{tabular}
    \label{tab:#2_units}
\end{table}%
}
\newcommand{\deftablevar}[3]{%
    $#1$ & $\si{#2}$ & #3 \\
}
\newcommand{\deftableobj}[3]{%
    $#1$ & \textit{#2} & #3 \\
}

\title{Le canon de Paris}
\date{2020-06-07}
\author{Ewen Le Bihan}

\begin{document}

\maketitle
\begin{abstract}
	Sujet disponible à \url{https://labolycee.org/le-canon-de-paris} ou dans ce dossier, à \texttt{./sujet.pdf}. Épreuve \texttt{19 PYSCOAS 1}
\end{abstract}


\section{Expulsion de l'obus}
\subsection{}

\[
	p = mv 
\]
Or, ni la masse $m$, ni la vitesse $v$ ne change, car:
\begin{description}
	\item[$m$] La masse du canon est invariante
	\item[$v$] Nous sommes dans un système pseudo-isolé, donc le centre d'inertie a une vitesse constante
\end{description}

Donc la quantité de mouvement $p$ ne change pas non plus.
\subsection{}

Nous sommes dans un système pseudo-isolé, donc:

\subsection{}
\begin{align*}
	\Sigma F &= m\vec a \\
\end{align*}

\section{}

\subsection{}

\subsection{}

\subsection{}

\section{}
\subsection{}
Pour déterminer la durée de vol, on cherche le temps pour lequel $y=0$
\begin{align*}
	y(t) &= 0 \\
             &=  -\frac{1}{2}gt^2+v_0+\sin\alpha \cdot t \\
	     &= \text{?} \\
	\iff t_1 &= 2\frac{v_0\sin\alpha}{g} \\
\end{align*}

Calcul de $t_1$ 

\begin{align*}
	t_1&= 2 \frac{\text{?}\sin\text{?}}{9.81} \\
	&= \SI{250}{\second} \\
\end{align*}

\subsection{}
On trouve l'altitude maximale en posant 
\begin{align*}
	v_y &= 0 \\
	\iff -9.8t + 1226 &= 0 \\
	\iff t_2 &= \SI{125}{\second} \\
\end{align*}

On calcule ensuite $y$ pour $t_2$ 

\begin{align*}
	y(t_2)&= -\frac{1}{2}g  \cdot 125^2 + 1600 \cdot \sin{50}  \cdot 125 \\
	&= \SI{76.5}{\kilo\gram} \\
\end{align*}

\subsection{}

frottements

\end{document}
