\documentclass{article}
\usepackage[utf8]{inputenc}
\usepackage[a4paper, total={6.5in, 9.5in}]{geometry}
\usepackage{hyperref}
\usepackage{amsmath, amssymb, dsfont}
\usepackage{mathtools}
\usepackage{chemfig}
\usepackage{chemformula}
\usepackage{xcolor}
\definecolor{gray07}{gray}{0.7}
\usepackage{float}
\usepackage{enumitem}
\usepackage{cancel}
\usepackage{pgfplots}
\usepgfplotslibrary{units} % to add units easily to axis
\usepgfplotslibrary{fillbetween} % to fill inbetween curves
\usepgfplotslibrary{colormaps} % to create colormaps
\pgfplotsset{width=12.2cm, height=7cm}
\pgfplotsset{compat=newest} %(making it only compatalbe with
%new releases of pgfplots)
\pgfdeclarehorizontalshading{visiblelight}{50bp}{
color(0.00000000000000bp)=(violet);
color(8.33333333333333bp)=(blue);
color(16.66666666666670bp)=(cyan);
color(25.00000000000000bp)=(green);
color(33.33333333333330bp)=(yellow);
color(41.66666666666670bp)=(orange);
color(50.00000000000000bp)=(red)
}%
\usepackage{tikz}
\usetikzlibrary{decorations.pathreplacing}
\usepackage{multirow}
\usepackage{caption}
\usepackage[load-configurations = abbreviations]{siunitx}
\sisetup{inter-unit-product=\ensuremath{{}\cdot{}}, exponent-product=\ensuremath{{}\cdot{}}, group-minimum-digits=3, range-phrase=-, range-units=single}
\usepackage{graphicx}
\graphicspath{ {./} }
\usetikzlibrary{shapes.arrows}
\let\ce\ch
\newcommand{\bit}{\text{bit}}
\newcommand{\equilibrium}[2]{$\ce{#1}\rightleftharpoons\ce{#2}$}
\newcommand{\soom}{\stackrel{10}{=}}
\newcommand{\vect}{\overrightarrow}
\newcommand{\formulaPyramidFactory}[4]{
	\begin{tikzpicture}[scale=1.25]
		\draw (0,0) -- (-0.5, -0.75) node  [align=center, fill=white, midway] {\footnotesize \---};
		\draw (0,0) -- (0.5, -0.75) node [align=center, fill=white, midway] {\footnotesize \---};
		\draw (0.5, -0.75) -- (-0.5, -0.75) node [align=center, fill=white, midway] {\footnotesize #4};

		\draw (0,0) node [above, fill=white] {#1};
		\draw (-0.5, -0.75) node [left, fill=white] {#2};
		\draw (0.5, -0.75) node [right, fill=white] {#3};
	\end{tikzpicture}
}
\newcommand{\formulaPyramid}[3]{\formulaPyramidFactory{#1}{#2}{#3}{$\times$}}
\newcommand{\formulaPyramidAdditive}[3]{\formulaPyramidFactory{#1}{#2}{#3}{$+$}}
\newenvironment{definitions}{\begin{description}[leftmargin=!,labelwidth=\widthof{\bfseries Lorem ipsum dolor}]}{\end{description}}
\newcommand{\deftable}[2]{%
%\hline
\begin{table}[h]
    \centering
    \begin{tabular}{llp{100mm}}%
        %& unité/type & Explication \\ \hline
        #1
    \end{tabular}
    \label{tab:#2_units}
\end{table}%
}
\newcommand{\deftablevar}[3]{%
    $#1$ & $\si{#2}$ & #3 \\
}
\newcommand{\deftableobj}[3]{%
    $#1$ & \textit{#2} & #3 \\
}

% =======================
% SECTIONS FORMATTING
% =======================
\renewcommand{\thesection}{}
\renewcommand{\thesubsection}{\arabic{subsection}}
\renewcommand{\thesubsubsection}{\alph{subsubsection})}
\makeatletter
\def\@seccntformat#1{\csname #1ignore\expandafter\endcsname\csname the#1\endcsname\quad}
\let\sectionignore\@gobbletwo
\let\latex@numberline\numberline
\def\numberline#1{\if\relax#1\relax\else\latex@numberline{#1}\fi}
\makeatother

% =======================
% METADATA
% =======================
\title{Synthèse organique}
\date{2020-05-20}
\author{Ewen Le Bihan}

\begin{document}

\maketitle

\section{@496 3}
\subsection{}

\begin{enumerate}
	\item Introduction des réactifs
	\item \emph{On adapte...} Chauffage à reflux (Réaction)
	\item \emph{On verse...} Extraction liquide-liquide (Isolement)
	\item \emph{On la soumet...} Distillation (Purification)
\end{enumerate}

\subsection{}

Quantité de matière

\subsection{}

La chaleur

\subsection{}

???

\subsection{}

\begin{itemize}
	\item Température d'ébullition
	\item Indice de réfraction
	\item Chromatographie
	\item Spectre à résonance magnétique nucléaire (RMN)
	\item Spectre infrarouge (IR)
\end{itemize}

\section{@496 5}

\subsection{}

\newcommand{\ethano}{\ce{H3CCOOH}}
\newcommand{\alcbenz}{\ce{C6H5CH2OH}}
Mélange équimolaire $\implies n_{\ethano} = n_{\alcbenz}$ 


\subsection{}

\begin{equation*}
	\begin{split}
		V &= \frac{15.8}{2} \\
		  &= \SI{7.6}{\milli\liter}
	\end{split}
\end{equation*}

\subsection{}
\subsubsection{}
??? 
%\begin{equation*}
%	\begin{split}
%		n_{\ethano} &= \frac{m}{M} \\
%			    &= \frac{10}{60} \\
%			    &= \SI{1.67}{\mol} \\
%	\end{split}
%\end{equation*}

\subsubsection{}
???
\subsubsection{}
Diverses raisons:

\begin{itemize}
	\item Pertes lors des manipulations
	\item Réaction non-totale
	\begin{itemize}
		\item Réaction non-totale
		\item Réaction non terminée
	\end{itemize}
\end{itemize}

\section{@398 7}

\begin{table}[h]
	\begin{tabular}{c|l}
		a & Essorage par filtration sous vide \\
		b & Extraction liquide-liquide \\
		c & Extraction liquide-liquide
	\end{tabular}
\end{table}

\section{@497 10}

\subsection{}

\begin{description}
	\item[athermique] Qui ne dégage pas ni n'absorbe de chaleur
	\item[exothermique] Qui dégage de la chaleur
\end{description}

\subsection{}

\begin{itemize}
	\item Faire fondre les réactifs solides
	\item Accélérer la réaction
\end{itemize}

\subsection{}


\begin{table}[h]
	\begin{tabular}{c|l}
		a & 1 \\
		b & 3 \\
		c & 2 \\
	\end{tabular}
\end{table}

\section{@498 12}

$c$ et $a$ sont sélectives.

\section{@499 16}

\subsection{}

Calculons d'abord la masse volumique $\rho_{\text{alcool}}$ de l'alcool

\[\begin{split}
	\rho_{ \text{alcool} } &= d_{ \text{alcool} } \cdot \rho_{ \text{eau} } \\
			       &= 0.79 \cdot 1 \\
			       &= \SI{0.79}{\kilo\gram\per\liter} \\
			       &= \SI{790}{\gram\per\liter}
\end{split}\]

Déterminons ensuite la masse:

\[\begin{split}
	m_{\text{alcool}} &= \rho_{\text{alcool}} \cdot V_{\text{alcool}} \\
			  &= 790 \cdot \SI{25e-3}{} \\
			  &= \SI{19.75}{\gram}
\end{split}\]

Puis, enfin, la quantité de matière $n_{\text{alcool}}$:

\[\begin{split}
	n_{\text{alcool}} &= \frac{m_{\text{alcool}}}{M_{\text{alcool}}} \\
			  &= \frac{19.75}{46} \\
			  &= \SI{4.3e-1}{\mol}
\end{split}\]

L'acide méthanoïque est introduit en quantité \emph{équimolaire}, donc:

\[
	n_{\text{méth}} &= n_{\text{alcool}} \\
\]

\[\begin{split}
	m_{\text{méth}} &= M_{\text{méth}} \cdot n_{\text{alcool}} \\
			&= 46 \cdot \SI{4.3e-1}{} \\
			&= \SI{19.78}{\gram}
\end{split}\]

\subsection{}
L'acide sulfurique sert à accélérer la réaction

\subsection{}
De haut en bas:

\begin{enumerate}
	\item Dépot organique
	\item Méthanoate d'éthyle
	\item Eau (salée)?
\end{enumerate}

\subsection{}

\[\begin{split}
	n_{\text{théo}} &= n_{\text{méth}} \\
	\iff m_{\text{théo}} &= M_{\text{méthanoate}} \cdot n_{\text{meth}} \\
			     &= 74 \cdot 0.43 \\
			     &= \SI{31.82}{\gram}
\end{split}\]

\subsection{}

\[\begin{split}
	R_1 &= \frac{m_{\text{exp}}}{m_{\text{théo}}} \\
	    &= \frac{21}{31.82} \\
	    &= 0.66 \quad\text{soit } 66\%
\end{split}\]

\subsection{}
\subsubsection{}
Distillation fractionnée

\subsubsection{}
Par rapport au montage à reflux, les pics de colonne de Vigreux
permettent à chaque espèce de condenser séparéement

\subsubsection{}
????

\subsection{}
\subsubsection{}

\[\begin{split}
	m_\text{esther} &= \rho_\text{esther} \cdot V_\text{esther} \\
			&= 1000 \cdot 0.91 \cdot \SI{31.5e-3}{} \\
			&= \SI{28.67}{\gram}
\end{split}\]

\subsubsection{}

\[\begin{split}
	R_2 &= \frac{m_\text{esther}}{m_\text{théo}} \\
	    &= \frac{28.67}{31.82} \\
	    &= 0.9 \quad\text{soit } 90\%
\end{split}\]

\subsubsection{}

\[\begin{split}
	\frac{90-66}{90} &= 0.27\quad\text{soit }26\%
\end{split}\]

$R_2 > R_1$, la nouvelle méthode s'est montrée efficace, augmentant le rendement de 26\%.

\end{document}
