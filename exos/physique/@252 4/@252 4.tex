\documentclass{article}
\usepackage[utf8]{inputenc}
\usepackage[a4paper, total={6.5in, 9.5in}]{geometry}
\usepackage{float}
\usepackage{amsmath}
\usepackage{amssymb}
\usepackage{mathtools}
\usepackage{tensor}
\newcommand{\torseur}[7]{
\tensor[_{#1}]{\left[ \begin{array}{cc}
    #2 & #5 \\
    #3 & #6 \\
    #4 & #7
\end{array} \right]}{_{(\vec{x};\vec{y};\vec{z})}}
}
\usepackage{siunitx}
\sisetup{output-decimal-marker={,},group-minimum-digits=4,abbreviations}
\sisetup{inter-unit-product=\ensuremath{{}\cdot{}}}
\newcommand{\deftable}[2]{%
%\hline
\begin{table}[h]
    \centering
    \begin{tabular}{llp{130mm}}%
        %& unité/type & Explication \\ \hline
        #1
    \end{tabular}
    \label{tab:#2_units}
\end{table}%
}
\newcommand{\deftablevar}[3]{%
    $#1$ & $\si{#2}$ & #3 \\
}
\newcommand{\deftableobj}[3]{%
    $#1$ & \textit{#2} & #3 \\
}

\title{4 p252}
\author{Ewen Le Bihan}
\date{2019-11-21}

\begin{document}

\section{}
\subsection{}

Le référentiel d'étude est géocentrique et est considéré Galiléen

\subsection{}

\begin{equation*}
  \begin{split}
    \vec a &= \frac{GM_T}{(R_T+h)^2} \vec n \\
  \end{split}
\end{equation*}

\subsection{}
D'après ce qui précède:

\begin{equation*}
  \begin{split}
    \vec a &= \frac{GM_T}{(R_T+h)^2} \vec n \\
           &= \frac{6.67\cdot10^{-11}\cdot5.98\cdot10^{24}}{(6380\cdot10^3+20180\cdot10^3)^2} \vec n \\
           &\approx 0 \vec n
  \end{split}
\end{equation*}

L'accélération $a$ est nulle, donc le mouvement circulaire d'un satellite est bien uniforme.

\section{}

\begin{equation*}
  \begin{split}
    v &= \sqrt{\frac{GM_T}{R_T+h}}
  \end{split}
\end{equation*}

\section{}

\begin{equation*}
  \begin{split}
    v &= \sqrt{\frac{GM_T}{R_T+h}} \\
      &= \sqrt{\frac{6.67\cdot10^{-11}\cdot5.98\cdot10^{24}}{6380\cdot10^3+20180\cdot10^3}} \\
      &\approx \SI{3875.96}{\meter\per\second} \\
      &\approx \SI{13951}{\kilo\meter\per\hour} \\
      &\approx \SI{14}{\mega\meter\per\hour}
  \end{split}
\end{equation*}

La vitesse est vérifée. Vérifions maintenant la période $T = \SI{12}{\hour}$

\begin{equation*}
  \begin{split}
    T &= \frac{\overbrace{2\pi (R_T+h)}^{\text{périmètre de l'orbite}}}{v} \\
    &= \frac{2\pi (6380+20180)}{14000} \\
    &\approx \SI{11.92}{\hour} \\
    &\approx \SI{12}{\hour}
  \end{split}
\end{equation*}

La période est également vérifée

\section{}

Une rotation de la Terre prend 24 heures, alors que le satellite parcours la même distance en seulement 12 heures. Du point de vue de la Terre, le satellite tourne $\frac{24}{12} = 2$ fois plus vite. Il n'est donc pas immobile d'un point de vue de la Terre, et n'est donc pas géostationnaire.

\end{document}
