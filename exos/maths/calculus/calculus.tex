\documentclass{article}
\usepackage[utf8]{inputenc}
\usepackage[a4paper, total={6.5in, 9.5in}]{geometry}
\usepackage{amsmath, amssymb}
\usepackage{chemfig}
\usepackage{chemformula}
\usepackage{float}
\usepackage{dsfont}
\usepackage{cancel}
\usepackage{pgfplots}
\usepackage{dsfont}
\usepackage{multirow}
\usepackage{caption}
\usepackage{calc}
\usepackage{hyperref}
\usepackage{enumitem}  
\usepackage{graphicx}
\usepackage{scalerel,stackengine}
\stackMath
\newcommand\reallywidehat[1]{%
\savestack{\tmpbox}{\stretchto{%
  \scaleto{%
    \scalerel*[\widthof{\ensuremath{#1}}]{\kern-.6pt\bigwedge\kern-.6pt}%
    {\rule[-\textheight/2]{1ex}{\textheight}}%WIDTH-LIMITED BIG WEDGE
  }{\textheight}% 
}{0.5ex}}%
\stackon[1pt]{#1}{\tmpbox}%
}
\parskip 1ex
\graphicspath{ {./} }
\usetikzlibrary{shapes.arrows}
\let\ce\ch
\newcommand{\im}{\text{Im}\,}
\newcommand{\re}{\text{Re}\,}
\newcommand{\img}{\text{Img}\,}
\newcommand{\R}{\mathds{R}}
\newcommand{\C}{\mathds{C}}
\newcommand{\N}{\mathds{N}}
\newcommand{\Z}{\mathds{Z}}
\newcommand{\conj}[1]{\overline{#1}}
\newcommand{\Aff}{\text{Aff}}
\newcommand{\oo}{\infty}
\newcommand{\twoRows}[1]{\multirow{2}{*}{#1}}
\newcommand{\threeRows}[1]{\multirow{3}{*}{#1}}
\newcommand{\twoCols}[1]{\multicolumn{2}{c|}{#1}}
\newcommand{\threeCols}[1]{\multicolumn{3}{|c|}{#1}}
\newcommand{\twoColsNB}[1]{\multicolumn{2}{c}{#1}}
\newcommand{\goesto}[2]{\xrightarrow[#1\:\to\:#2]{}}
\newcommand{\liminfty}{\lim_{x\to+\oo}}
\newcommand{\limminfty}{\lim_{x\to-\oo}}
\newcommand{\limzero}{\lim_{x\to0}}
\newcommand{ \const}{\text{cste}}
\newcommand{\et}{\:\text{et}\:}
\newcommand{\ou}{\:\text{ou}\:}
\newcommand{\qed}{\square}
\newcommand{\placeholder}{\diamond}
\newcommand{\mediateur}{\:\text{med}\:}
\newcommand{\milieu}{\:\text{mil}\:}
\newcommand{\vect}[1]{\overrightarrow{#1}}
\newenvironment{descriptiona}{\begin{description}[leftmargin=!,labelwidth=\widthof{\bfseries The longest label}]}{\end{description}}
\renewcommand{\arraystretch}{1.4}
\newcommand{\point}[2]{(#1;\;#2)}
\newcommand{\spacepoint}[3]{\begin{pmatrix}#1\\#2\\#3\end{pmatrix}}
\newcommand{\spliteq}[1]{\begin{equation*}\begin{split}#1\end{split}\end{equation*}}
\renewcommand{\leq}{\leqslant}
\renewcommand{\geq}{\geqslant}


\title{Exercices: Calculus}
\author{Ewen Le Bihan}
\date{2020-05-15}

\begin{document}
\maketitle

\begin{center}
	\emph{Tout les exercices ici proviennent du livre "Calculus" disponible 
	\href{https://www.amazon.fr/Calculus-\%C3\%A9dition-approfondir-connaissances-math\%C3\%A9matiques/dp/2820807178/}{\underline{sur amazon}}}
\end{center}
\section{} %difficulty=1

Montrons par récurrence:

\[
	\forall n \in \N^\ast, \quad \sum_{i=1}^{n} i^3 = \left( \frac{n(n+1)}{2} \right)^2
\]

Pour $n$ dans $\N^\ast$, on note la propriété $\mathcal{P}_n$:

\[
	1^3 + 2^3 + \dots + n^3 = \left( \frac{n(n+1)}{2} \right)^2	
\]

\paragraph{Initialisation.} Montrons que, pour $n = 1$, $\mathcal{P}_n$ est vraie:

\[
	\left(\frac{1(1+1)}{2}\right)^2 = 1 = 1^2
\]

$\mathcal{P}_n$ est donc vraie.

\paragraph{Hérédité.} Soit $n$ dans $\N^\ast$ tel que $\mathcal{P}_n$ est vraie. 
On a donc:

\spliteq{
	1^3 + 2^3 + \dots + n^3 + (n+1)^3 &= \left( \frac{n(n+1)}{2} \right)^2 + (n+1)^3 \\
					  &= \frac{(n(n+1))^2 + 4(n+1)^3}{4} \\
					  &= \frac{n^2(n^2+2n+1) + 4(n^3+3n^2+3n+1)}{4} \\
					  &= \frac{n^4+2n^3+n^2+4n^3+12n^2+12n+4}{4} \\
					  &= \frac{n^4+6n^3+13n^2+12n+4}{4}
}

Or

\spliteq{
	\left( \frac{(n+1)(n+2)}{2} \right)^2 &= \frac{(n+1)^2(n+2)^2}{4} \\
					      &= \frac{n^4+6n^3+3n^2+12n+4}{4}
}

\newpage
En fin de compte, on obtient $\mathcal{P}_{n+1}$:

\spliteq{
	1^3 + 2^3 + \dots + n^3 + (n+1)^3 &= \frac{n^4+6n^3+3n^2+12n+4}{4} \\
					  &= \left( \frac{(n+1)(n+2)}{2} \right)^2
}

$\mathcal{P}_n$ est donc initialisée et héréditaire, 
donc $\mathcal{P}_n$ est vraie pour tout $n$ dans $\N^\ast$ $\qed$


\section{} %difficulty=2

Montrons par récurrence

\[
	\forall x \in \R, \forall n \in \N, \quad |\sin{(nx)}| \leq n|\sin{x}|
\]

Soit, pour tout $n$ dans $\N$, $\mathcal{P}_n$ la propriété suivante:

\[
	|\sin(nx)| \leq n|\sin x|
\]

\paragraph{Initialisation.} Pour $n = 0$, montrons que $\mathcal{P}_n$ est vraie.
On a, pour tout $x$ dans $\R$:

\spliteq{
	|\sin(0x)| = 0 &\land 0|\sin x| = 0 \\
	\implies |\sin(0x)| &\leq 0|\sin x|
}

$\mathcal{P}_0$ est donc vraie.

\paragraph{Hérédité.} Prouvons que, pour tout entier naturel $n$ et tout réel $x$, $\mathcal{P}_n$ est vraie. 


\spliteq{
	\sin((n+1)x) &= \sin(nx+x) \\
		     &= \sin(nx)\cos x + \cos(nx) \sin x \\
	\iff |\sin((n+1)x) | &= | \sin(nx)\cos x + \cos(nx) \sin x | \\
			     &\leq | \sin(nx)||\cos x|+|\cos(nx)||\sin x | \\
}

Remarquons que $\cos x \leq 1 \implies |\cos x| \in [0, 1]$

\spliteq{
	|\sin((n+1)x)| &\leq |\sin(nx)| + |\sin x| \\
		       &\leq n|\sin x| + |\sin x| \\
		       &\leq (n+1)|\sin x|
}

\section{}

\spliteq{
	u_1 &= u_0^2 = u_0^2\\
	u_2 &= (u_0^2)^2 = u_0^4 \\
	u_3 &= ((u_0^2)^2)^2 = u_0^8\\
	u_4 &= (((u_0^2)^2)^2)^2 = u_0^{16} \\
}

On conjecture que $u_n = u_0^{2^n}$.
Soit, pour tout $n$ dans $\N$, $\mathcal{P}_n$ la propriété $u_n = u_0^{2^n}$.

\paragraph{Initialisation.} Avec $n=0$:

\[
	u_0^{2^0} = u_0^1 = u_0
\]

$\mathcal{P}_0$ est donc validée.


\paragraph{Hérédité.} Admettons que, pour un certain $n \in \N$, $\mathcal{P}_{n}$ est vraie. On cherche à montrer que $\mathcal{P}_{n+1}$ l'est également.

\spliteq{
	u_{n+1} &= (u_0^{2^n})^2 \\
		&= u_0^{2^n\cdot2} \\
		&= u_0^{2^{n+1}}
}

$\mathcal{P}_{n+1}$ est donc validée, $\mathcal{P}_{n}$ est initialisée et héréditaire donc vraie, ce qui valide également la conjecture, autrement dit: $u_n = u_n^{2^{n+1}}$ $\qed$


\section{} %difficulty=3

\section{} %difficulty=2

\section{} %difficulty=2

À chaque transmission, la probabilité que le message soit correct est $p$. Autrement dit,
pour le premier opérateur, il y a une transmission, pour le deuxième deux transmissions,
et pour le $n$-ième $n$ transmissions.

Soit $(u_n)_{n>0}$ une suite telle que $u_n$ représente la probabilité que le message
soit reçu correctement par le $n$-ième opérateur. L'objectif est de trouver la définition
de cette suite.

Deux erreurs de transmissions font subir au bit deux inversions, ce qui est équivalent 
au fait de n'avoir eu aucune erreur ($1 \to 0 \to 1 \iff 1 \to 1 \to 1$). 
On peut facilement généraliser: il faut que le nombre d'erreurs de transmissions 
subies soit \emph{pair}.

Le nombre de transmissions correctes n'a pas d'importance: une transmission correcte 
n'engendre pas de modification du signal ($1 \to 1 \to 1 \iff 1 \to 1$)

Explorons les expressions de $u_n$ pour $n \in \{1, 2, 3, 4\}$

On pose $q = 1 - p$

\begin{align*}
	u_1 &= p \\
	u_2 &= p^2 + q^2 \\
	u_3 &= p^3 + pqq + qqp + qpq = p^3 + 3q^2p \\
	u_4 &= p^4 + qqpp + pqqp + ppqq + qpqp + pqpq + qppq + q^4 = p^4 + 6q^2p^2 + q^4 \\
\end{align*}

\begin{align*}
	i &= x\mapsto \begin{cases}
		0 & \text{si} \mod(x, 2) \neq 0 \\
		1 & \text{sinon}
	\end{cases} \\
	u_n &= n \mapsto p^n + (2^n-i(n))q^{\frac{2^n}{2}}p^{\frac{2^n}{2}} + i(n)q^n \\
\end{align*}

\section{} %difficulty=2
\begin{align*}
	f \circ f &= x \mapsto  \frac{\frac{x}{\sqrt{1+cx^2} }}{\sqrt{1+c\frac{x}{\sqrt{1+cx^2} }} } \\
\end{align*}

\section{} %difficulty=1

\paragraph{Initialisation} Pour tout $n \in \N$, soit $P_n$ l'assertion $u_n = 2^{n} + 3^{n}$.

\begin{align*}
	u_0 &= 2 = 2^{0} + 3^{0} \implies P_0\quad\text{est vraie} \\
	u_1 &= 5 = 2^{1} + 3^{1} \implies P_1\quad\text{est vraie} \\
\end{align*}

\paragraph{Hérédité} On suppose que, pour un certain $n \in \N$, $P_n$ et $P_{n+1}$ sont vraies. Montrons qu'alors $P_{n+2}$ est aussi vraie.

\begin{align*}
	u_{n+2} &= 5u_{n+1} - 6u_n \\
		&= 5(2^{n+1}+3^{n+1}) - 6(2^{n}+3^{n}) \\
		&= (2+3)(2^{n+1}+3^{n+1}) - 6(2^{n}+3^{n}) \\
		&= 2^{n+2}+2 \cdot 3^{n+1} + 3 \cdot 2^{n+1} + 3^{n+2} - 6 \cdot 2^{n} - 6 \cdot 3^{n} \\
		&= 2^{n+2} + \cancel{2 \cdot 3 \cdot 3^{n}} + \cancel{3 \cdot 2 \cdot 2^{n}} + 3^{n+2} - \cancel{3 \cdot 2 \cdot 2^{n}} - \cancel{2 \cdot 3 \cdot 3^{n}}\\
		&= 2^{n+2} + 3^{n+2} \\
		&\implies P_{n+2}\;\text{est vraie}
\end{align*}

\section{} %difficulty=2
\begin{align*}
	u_2 &= \frac{u_1^2}{u_0} = \frac{4}{1} = 4 \\
	u_3&= \frac{u_2^2}{u_1} = \frac{16}{2} = 8 \\
	u_4&= \frac{u_3^2}{u_2} = \frac{64}{4} = 16 \\
\end{align*}

\paragraph{Initialisation} Pour tout $n \in \N$, soit $P_n$ l'assertion $u_n = 2^n$.

\begin{align*}
	u_0 &= 1 = 2^{0}  \\
	u_1&= 2 = 2^{1} \\
\end{align*}

\paragraph{Hérédité} On considère que, pout un certain $n \in \N$, $P_{n}$ et $P_{n+1}$ sont vraies. Montrons qu'alors $P_{n+2}$ est aussi vraie.
\begin{align*}
	u_{n+2} &= \frac{u_{n+1}^2}{u_n} \\
		&= \frac{(2^{n+1})^2}{2^{n}} \\
		&= \frac{2 \cdot \cancel{2^n} \cdot 2 \cdot 2^{n}}{\cancel{2^{n}}} \\
		&= 2 \cdot 2 \cdot 2^{n} \\
		&= 2^{n+2} \\
		&\implies P_{n+2}\quad\text{est vraie}
\end{align*}

\section{} %difficulty=3
\section{} %difficulty=2

Soit, pour tout $n$ dans $\N$, $\lambda_n = \frac{\alpha^{n}}{\sqrt{5} }$.

Si la différence entre $\lambda_n$ et $F_n$ est inférieure à 0.5, il n'y a aucun entier plus proche de $\lambda_n$.
On cherchera donc à prouver l'assertion $P_n$: Soit, pour tout $n \in \N$, $P_n$ l'assertion $|\lambda_n-F_n|<0.5$.

% \paragraph{Initialisation} À $n = 0$
% \begin{align*}
% 	|\lambda_0 - F_0| &= |\frac{\alpha^{0}}{\sqrt{5} } - 0| \\
% 	&= |\frac{1}{\sqrt{5} }|\\
% 	&< 0.5\quad\text{car $\sqrt$ est croissante, $\sqrt 4=2$ et $5 > 4$} \\
% 	&\implies P_0\quad\text{est vraie}
% \end{align*}
% 
% \paragraph{Hérédité} On suppose que, pour un certain $n \in \N$, $P_n$ est vraie. Montrons qu'alors $P_{n+1}$ est aussi vraie.
% 
% \begin{align*}
% 	|\lambda_{n+1} - F_{n+1}| &= \left|\frac{\alpha^{n+1}}{\sqrt{5} } - F_{n+1}\right| \\
% 	              &= \left|\frac{\alpha^{n}}{\sqrt{5} } \cdot \frac{\alpha}{\sqrt{5} }-(F_{n-1}+F_n)\right| \\
% 
% \end{align*}

\begin{align*}
	| \frac{\alpha^{n}}{\sqrt{5} } - F_n | &= | \frac{\alpha^{n}}{\sqrt{5} } - \frac{\alpha^{n}-\beta^{n}}{\sqrt{5} } \\
					       &= \left| \cancel{\frac{\alpha^{n}}{\sqrt{5} }-\frac{\alpha^{n}}{\sqrt{5} }} - \frac{\beta^{n}}{\sqrt{5} } \right|\\
					       &= \frac{\beta^{n}}{\sqrt{5} } \\
					       &= \left(\frac{1-\sqrt{5} }{2}\right)^{n} \cdot \frac{1}{\sqrt{5} }\\
					       &= \left( \frac{1}{2} - \frac{\sqrt{5}}{2} \right)^{n} \frac{1}{\sqrt{5} } \\
\end{align*}

Or, comme

\begin{align*}
	\frac{1}{2} &> \frac{1}{2}-\frac{\sqrt{5} }{2} \\
		    &> \left( \frac{1}{2}-\frac{\sqrt{5} }{2} \right) ^{n}\quad\text{car $n > 0$} \\
\end{align*}

et

\begin{align*}
	\frac{1}{\sqrt{5} } &< \frac{1}{2}
\end{align*}

alors 

\begin{align*}
	\left( \frac{1}{2} - \frac{\sqrt{5}}{2} \right)^{n} \frac{1}{\sqrt{5} } &< \frac{1}{2} \\
	\implies | \frac{\alpha^{n}}{\sqrt{5} } - F_n | &< 0.5
\end{align*}.

donc, pour $n \in \N$, $F_n$ est l'entier le plus proche de $\frac{\alpha^{n}}{\sqrt{5} }$

\section{} %difficulty=3
\section{} %difficulty=3
\section{} %difficulty=4
\section{} %difficulty=2

\subsection{}

\begin{align*}
	(x^2+y^2)(x'^2+y'^2)&= x^2x'^2+x^2y'^2+y^2x'^2+y^2y'^2 \\
	(x x' - yy')^2+(xy'+yx')^2 &= x^2x'^2\cancel{-2(x'xy'y)}+yy'^2+x^2y'^2\cancel{+2(x'xy'y)}+y'^2x^2 \\
	&= x^2x'^2+x^2y'^2+y^2x'^2+y^2y'^2 \\
\end{align*}

Donc

\begin{align*}
	(x^2+y^2)(x'^2+y'^2)&=(x x' - yy')^2+(xy'+yx')^2
\end{align*}

\subsection{}

Soit pour tout $k$ dans $\N^{\ast}$ $P_k$ l'assertion $e_1 \times \ldots \times e_k = (u^2+v^2)(u'^2+v'^2)$

\paragraph{Initialisation} $k=1$. L'expression est $e_1$, initialisation triviale.

\paragraph{Hérédité} On suppose que, pour un certain $k$ dans $\N^\ast$, $P_n$ est vraie. Montrons qu'alors $P_{n+1}$ est aussi vraie.

\begin{align*}
	e_1 \times \ldots \times e_k \times e_{k+1} &= (u^2+v^2)(u'^2+v'^2) \\
\end{align*}

% don't delete dis, baka ↓
\end{document}

