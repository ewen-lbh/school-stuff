\documentclass{article}
\usepackage[utf8]{inputenc}
\usepackage[a4paper, total={6.5in, 9.5in}]{geometry}
\usepackage[bookmarks,hidelinks]{hyperref}
\usepackage{amsmath, amssymb}
\usepackage{chemformula}
\usepackage{float}
\usepackage{cancel}
\usepackage{multirow}
\usepackage{caption}
\usepackage{calc}  
\usepackage{enumitem}  
\usepackage{graphicx}
\usepackage{scalerel,stackengine}
\stackMath
\newcommand\reallywidehat[1]{%
\savestack{\tmpbox}{\stretchto{%
  \scaleto{%
    \scalerel*[\widthof{\ensuremath{#1}}]{\kern-.6pt\bigwedge\kern-.6pt}%
    {\rule[-\textheight/2]{1ex}{\textheight}}%WIDTH-LIMITED BIG WEDGE
  }{\textheight}% 
}{0.5ex}}%
\stackon[1pt]{#1}{\tmpbox}%
}
\parskip 1ex
\graphicspath{ {./} }
\usetikzlibrary{shapes.arrows}
\let\ce\ch
\newcommand{\im}{\text{Im}\,}
\newcommand{\re}{\text{Re}\,}
\newcommand{\img}{\text{Img}\,}
\newcommand{\R}{\mathbb{R}}
\newcommand{\C}{\mathbb{C}}
\newcommand{\N}{\mathbb{N}}
\newcommand{\Z}{\mathbb{Z}}
\newcommand{\conj}[1]{\overline{#1}}
\newcommand{\Aff}{\text{Aff}}
\newcommand{\oo}{\infty}
\newcommand{\twoRows}[1]{\multirow{2}{*}{#1}}
\newcommand{\threeRows}[1]{\multirow{3}{*}{#1}}
\newcommand{\twoCols}[1]{\multicolumn{2}{c|}{#1}}
\newcommand{\threeCols}[1]{\multicolumn{3}{|c|}{#1}}
\newcommand{\twoColsNB}[1]{\multicolumn{2}{c}{#1}}
\newcommand{\goesto}[2]{\xrightarrow[#1\:\to\:#2]{}}
\newcommand{\liminfty}{\lim_{x\to+\oo}}
\newcommand{\limminfty}{\lim_{x\to-\oo}}
\newcommand{\limzero}{\lim_{x\to0}}
\newcommand{ \const}{\text{cste}}
\newcommand{\et}{\:\text{et}\:}
\newcommand{\ou}{\:\text{ou}\:}
\newcommand{\placeholder}{\diamond}
\newcommand{\mediateur}{\:\text{med}\:}
\newcommand{\milieu}{\:\text{mil}\:}
\newcommand{\vect}[1]{\overrightarrow{#1}}
\newenvironment{descriptiona}{\begin{description}[leftmargin=!,labelwidth=\widthof{\bfseries The longest label}]}{\end{description}}
\renewcommand{\arraystretch}{1.4}
\newcommand{\point}[2]{(#1;\;#2)}
\newcommand{\spacepoint}[3]{\begin{pmatrix}#1\\#2\\#3\end{pmatrix}}

\title{Exercices: Trinômes du second degré}
\author{Ewen Le Bihan}
\date{2020-06-07}

\begin{document}
\maketitle

\abstract{}
Chaque section est un exercice, le nom de la section représente a le format suivant:
\texttt{@<page> <exercice>}, avec \texttt{<exercice>} le numéro de l'exercice et \texttt{<page>} le numéro de la page.

Par défaut, les exercices du livre le plus courant sont assumés, mais l'on peut préciser le livre avec la syntaxe suivante:

\texttt{<livre>@<page> <exercice>}.
Exemple:

\texttt{Sésamath@218 24} représente l'exercice numéro \emph{24} du livre \emph{Sésamath} à la page \emph{218}

\section{Calculus@60 110}

On a:
\begin{align*}
	f(x) &= (x^2-3)(1-\sqrt{x} )(|x|-6)(|4x+3|) \\
.\end{align*}



\section{Calculus@60 113}
\subsection{} \label{113 a)}
%TODO définir l'ensemble!!
\begin{align*}
	\sqrt{x-1} -\sqrt{2x-3} &\gtreqless 0 \\
	\iff \sqrt{x-1} &\gtreqless \sqrt{2x-3} \\
	\iff x-1 &\gtreqless 2x-3 \\
	\iff x &\gtreqless 2x-2 \\
	\iff -x &\gtreqless -2 \\
	\iff x &\gtreqless_\text{inv} 2
.\end{align*}
\subsection{}
Les quantités sont définies pour tout $x \in \R$.
On sépare les cas:
\begin{itemize}
	\item quand $2x-3 \ge  0$
	\item quand $x-1 \ge  0 \implies x \ge  1$ 
	\item quand ils ne le sont pas.
\end{itemize}
Quand $2x-3\ge 0$, les solutionns sont les mêmes que pour \ref{113 a)}.
Quand $x-1\ge 0$
\begin{align*}
	\sqrt{x-1} -\sqrt{2x-3} &\gtreqless 0 \\
	\iff \sqrt{x-1} &\gtreqless \sqrt{2x-3} \\
	\iff x-1 &\gtreqless 2x-3 \\
.\end{align*}

\end{document}
