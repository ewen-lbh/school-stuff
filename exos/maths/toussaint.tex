\documentclass{article}
\usepackage[utf8]{inputenc}
\usepackage[a4paper]{geometry}
\usepackage{amsmath, amssymb}
\usepackage{hyperref}
\usepackage{soul}
\hypersetup{
	    colorlinks=true,
		urlcolor=blue,
}
\usepackage{xcolor}

\newcommand{\R}{\mathbb{R}}
\newcommand{\id}{\operatorname{id}}
\newcommand{\cC}{\mathcal{C}}
\newcommand{\dD}{\mathcal{D}}
\newcommand{\fF}{\mathcal{F}}

\begin{document}
\pagestyle{empty}

\section*{Toussaint -- Dérivée \#2}

\[
	\id \stackrel{\text{def}}{=} \begin{cases}
		\R_+^\ast &\to \R_+^\ast \\
		x &\mapsto x
	\end{cases}
\]

On a:

\[
	\begin{cases}
		{\color{blue}\id} \in \fF(\R_+^\ast, \R_+^\ast) \\
		{\color{red}\id} \in \fF(\R_+^\ast, \R) & \text{car}\ \R_+^\ast \subseteq \R
	\end{cases}
\] 

Donc, d'après la définition 9 du \href{http://mpsi.daudet.free.fr/maths/polys/cours/FonctionsUsuelles/Fonctions_usuelles.pdf}{\underline{poly des fonctions usuelles}}:

\[
	{\color{blue} \id^{\color{red}{\id}}} = \exp \circ ({\color{red}\id} \cdot (\ln \circ {\color{blue}\id})) = \exp \circ (\ln \cdot \id)
\] 

De plus,

\[
	\begin{cases}
		\exp &\in \cC(\R, \R_+^\ast) \\
		\id \cdot \ln &\in \cC(\R_+^\ast, \R)
	\end{cases}
\] 

D'après le théorème de dérivation des fonctions composées:

\[
	\begin{cases}
		\exp \circ (\ln \cdot \id) &\in \dD(\R_+^\ast, \R_+^\ast) \\
		(\exp\circ(\ln \cdot \id))' &= (\ln \cdot \id)' \cdot \exp' \circ (\ln \cdot \id)
	\end{cases}
\] 
Finalement:

\begin{align*}
	(\id^{\id})' &= (\exp\circ(\ln \cdot \id))' \\
	&= (\ln \cdot \id)' \cdot \exp' \circ (\ln \cdot \id) \\
	&= (\ln' \cdot \id + \id' \cdot \ln) \cdot \exp\circ(\ln \cdot \id) \\
	&= \left(\frac{\id}{\id} + 1 \cdot \ln\right) \cdot  \exp\circ(\ln \cdot \id)\\
	&= (1+\ln) \cdot  \exp\circ(\ln \cdot \id)\\
\end{align*}

Pour ceux qui pensent que la notation avec $x\mapsto $ est mieux \st{ils ont tort}:

\[
	(x\mapsto x^x)' = x\mapsto (1+\ln x)e^{x \ln x}
\] 
\vfill
\begin{center}
	
{\tiny \href{https://open.spotify.com/artist/6tUc6r8aNeiiT1mElcnMx9?si=dIYSUMDrT_uH0d0yJEZnrw}{\underline{Faites-moi des vues sur Spotify}}}

\end{center}
\end{document}
