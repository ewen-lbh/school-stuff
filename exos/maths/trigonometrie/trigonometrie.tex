\documentclass{article}
\usepackage[utf8]{inputenc}
\usepackage[a4paper, total={6.5in, 9.5in}]{geometry}
\usepackage[bookmarks,hidelinks]{hyperref}
\usepackage{amsmath, amssymb}
\usepackage{chemformula}
\usepackage{float}
\usepackage{cancel}
\usepackage{multirow}
\usepackage{caption}
\usepackage{calc}  
\usepackage{enumitem}  
\usepackage{graphicx}
\usepackage{scalerel,stackengine}
\stackMath
\newcommand\reallywidehat[1]{%
\savestack{\tmpbox}{\stretchto{%
  \scaleto{%
    \scalerel*[\widthof{\ensuremath{#1}}]{\kern-.6pt\bigwedge\kern-.6pt}%
    {\rule[-\textheight/2]{1ex}{\textheight}}%WIDTH-LIMITED BIG WEDGE
  }{\textheight}% 
}{0.5ex}}%
\stackon[1pt]{#1}{\tmpbox}%
}
\parskip 1ex
\graphicspath{ {./} }
\usetikzlibrary{shapes.arrows}
\let\ce\ch
\newcommand{\im}{\text{Im}\,}
\newcommand{\re}{\text{Re}\,}
\newcommand{\img}{\text{Img}\,}
\newcommand{\R}{\mathbb{R}}
\newcommand{\C}{\mathbb{C}}
\newcommand{\N}{\mathbb{N}}
\newcommand{\Z}{\mathbb{Z}}
\newcommand{\conj}[1]{\overline{#1}}
\newcommand{\Aff}{\text{Aff}}
\newcommand{\oo}{\infty}
\newcommand{\twoRows}[1]{\multirow{2}{*}{#1}}
\newcommand{\threeRows}[1]{\multirow{3}{*}{#1}}
\newcommand{\twoCols}[1]{\multicolumn{2}{c|}{#1}}
\newcommand{\threeCols}[1]{\multicolumn{3}{|c|}{#1}}
\newcommand{\twoColsNB}[1]{\multicolumn{2}{c}{#1}}
\newcommand{\goesto}[2]{\xrightarrow[#1\:\to\:#2]{}}
\newcommand{\liminfty}{\lim_{x\to+\oo}}
\newcommand{\limminfty}{\lim_{x\to-\oo}}
\newcommand{\limzero}{\lim_{x\to0}}
\newcommand{ \const}{\text{cste}}
\newcommand{\et}{\:\text{et}\:}
\newcommand{\ou}{\:\text{ou}\:}
\newcommand{\placeholder}{\diamond}
\newcommand{\mediateur}{\:\text{med}\:}
\newcommand{\milieu}{\:\text{mil}\:}
\newcommand{\vect}[1]{\overrightarrow{#1}}
\newenvironment{descriptiona}{\begin{description}[leftmargin=!,labelwidth=\widthof{\bfseries The longest label}]}{\end{description}}
\renewcommand{\arraystretch}{1.4}
\newcommand{\point}[2]{(#1;\;#2)}
\newcommand{\spacepoint}[3]{\begin{pmatrix}#1\\#2\\#3\end{pmatrix}}

\title{Exercices: Trigonométrie}
\author{Ewen Le Bihan}
\date{2020-08-02}

\begin{document}
\maketitle

\abstract{}
Chaque section est un exercice, le nom de la section représente a le format suivant:
\texttt{@<page> <exercice>}, avec \texttt{<exercice>} le numéro de l'exercice et \texttt{<page>} le numéro de la page.

Par défaut, les exercices du livre le plus courant sont assumés, mais l'on peut préciser le livre avec la syntaxe suivante:

\texttt{<livre>@<page> <exercice>}.
Exemple:

\texttt{Sésamath@218 24} représente l'exercice numéro \emph{24} du livre \emph{Sésamath} à la page \emph{218}

\section*{}

\section{Calculus@50 65} %difficulty=1
\begin{align*}
	\cos\left( \frac{\pi}{3} - \frac{\pi}{4} \right) &= \cos\frac{\pi}{3} \cos\frac{\pi}{4} + \sin\frac{\pi}{3} \sin\frac{\pi}{4} \\
							 &= \frac{1}{2} \frac{\sqrt{2} }{2} + \frac{\sqrt{3} }{2} \frac{\sqrt{2} }{2} \\
							 &= \frac{\sqrt{2} }{4} + \frac{\sqrt{3} \sqrt{2} }{4} \\
							 &= \frac{\sqrt{2} + \sqrt{3} \sqrt{2} }{4} \\
							 &= \frac{\sqrt{6} - \sqrt{2} }{4} \\
\end{align*}

\begin{align*}
	\sin\left( \frac{\pi}{3}-\frac{\pi}{4} \right) &= \sin\frac{\pi}{3}\cos\frac{\pi}{4}-\sin\frac{\pi}{4}\cos\frac{\pi}{3} \\
	&= \frac{\sqrt{2} }{2} \frac{\sqrt{3} }{2} - \frac{1}{2}\frac{\sqrt{2} }{2} \\
	&= \frac{\sqrt{2} \sqrt{3} - \sqrt{2} }{4} \\
	&= \frac{\sqrt{6} -\sqrt{2} }{4} \\
\end{align*}

\section{Calculus@50 66} %difficulty=1

\section{Calculus@50 67} %difficulty=2

$ \sin$ a pour maximum 1. Or, d'après la formule de duplication pour $ \sin$, $ \sin(2x)=\mathbf{2}\cos x\sin x$. Donc $\cos x \sin x$ a pour maximum dans $\R$ $\frac{1}{2}$.

\section{Calculus@50 68} %difficulty=2

\section{Calculus@50 69} %difficulty=1

On pose $y = 2x$

\begin{align*}
	\cos(3x) &= \cos(x+y) \\
	  	&= \cos x \cos y - \sin x \sin y \\
		  &= \cos x \cos(2x) - \sin x \sin(2x) \\
		  &= \cos x 2\cos^2x-1 - \sin x 2\sin x \cos x\\
.\end{align*}

Or $ \sin x = \cos\left( \frac{\pi}{2}-x \right) $

\begin{align*}
	\cos (3x)   &= \cos x 2\cos^2x-1 - \sin x 2\sin x \cos x\\
		    &= \cos x 2 \cos^2x - 1 - \cos(\frac{\pi}{2}-x) 2 \cos(\frac{\pi}{2}-x) \cos x \\
		    &= \cos x 2 \cos ^2x-1 - 3\left( \cos \frac{\pi}{2} \cos x + \sin \frac{\pi}{2} \sin x \right) \cos x   \\
		    &= \cos x 2 \cos^2x-1 - 3\left( 0  \cos x + 1 \sin x \right) \cos x \\
		    &= \cos x 2 \cos^2x-1-3\sin x \cos x  \\
		    &= \cos x 2 \cos^2x-1-3\cos\left( \frac{\pi}{2}-x \right) \cos x \\
		    &=  \text{fuck this shit} \\
.\end{align*}

\section{Calculus@51 70} %difficulty=3

Soit, pour tout $n \in \N^{\ast}$, l'assertion $P_n$: $u_n = 2\cos\left( \frac{\pi}{2^{n+1}} \right) $.

\paragraph{Initialisation} Pour $n = 1$

\begin{align*}
	2\cos\left( \frac{\pi}{2^{2}} \right) &= 2\cos\frac{\pi}{4} \\
	&= 2 \frac{\sqrt{2} }{2}  \\
	&= \sqrt{2}  \\
.\end{align*}

\begin{align*}
	u_1 &= \prod_{k=1}^{1} u_1  \\
	&= u_1 \\
	&= \sqrt{2}  \\
.\end{align*}

$P_1$ est donc vraie.

\paragraph{Hérédité} Supposons que, pour un certain $n \in \N^{\ast}$, $P_n$ est vraie. Montrons qu'alors $P_{n+1}$ est également vraie.
\begin{align*}
	u_{n+1} &= \sqrt{2+u_n}  \\
		&= \sqrt{2+2\cos\left( \frac{\pi}{2^{n+1}} \right) }  \\
		&= \sqrt{2+\text{fuck this shit}}  \\
.\end{align*}

\section{Calculus@51 71} %difficulty=3
\subsection{}

\section{Calculus@52 72} %difficulty=1
\subsection{}
\begin{align*}
	\cos x &= \frac{1}{2} \\
	       &= \cos \frac{\pi}{3} \\
	\iff x &\equiv \pm \frac{\pi}{3}\quad[2\pi] \\
.\end{align*}

\subsection{}
\begin{align*}
	\sin(2x) &= \frac{\sqrt{2} }{2}  \\
		 &= \sin \frac{\pi}{4} \\
	\iff 2x &\equiv \frac{\pi}{4} \quad[2\pi]\quad\text{ou}\quad \frac{\pi}{4}+\pi\quad[2\pi]\\
	\iff x &\equiv \frac{\pi}{8}\quad [2\pi] \quad\text{ou}\quad \frac{\pi}{8} + \pi\quad[2\pi] 
.\end{align*}

\section{Calculus@52 73} %difficulty=2

\[
	\left[ \frac{\pi}{4}, 0 \right] \cup \left[ \frac{5\pi}{4}, 2\pi \right]
.\]

\section{Calculus@52 74} %difficulty=1

Pour tout $x \in  \R$, 

\begin{align*}
	\cos x \in [0, 1] &\implies \cos x \ge 0 \\
			  &\implies ( \cos^2x \ge 0 \\
			  &\land 4\cos x \ge 0 ) \\
			  &\implies \cos^2x + 4\cos x \ge 0 \\
			  &\implies \cos^2x + 4\cos x + 1 \ge  0 \\
\end{align*}

Or:

\begin{align*}
	2\cos^2x + 4\cos x + 1 &= 2\cos^2x - 2 + 3 + 4\cos x  \\
			       &= 2(\cos^2x - 1) + 4\cos x + 3  \\
			       &= 2\cos(2x)+4\cos x + 3  \\
.\end{align*}

Donc

\[
	2\cos(2x)+4\cos x + 3 \ge 0
.\] 

%TODO: pour quelles valeurs de x \in \R 2cos(x)+4cos(x)+3 = 0 ?

\section{Calculus@52 75} %difficulty=2

\begin{align*}
	a\cos x + b\sin x &= \sqrt{a^2+b^2} \cos(x-\phi) \\
			  &= \text{((WIP))} \\
.\end{align*}


\section{Calculus@52 76} %difficulty=2

\begin{align*}
	\cos x + \sin x &= \sqrt{\frac{3}{2}}  \\
	\iff (\cos x + \sin x)^2 &= \frac{3}{2} \\
	\iff \cos^2x + 2\cos x \sin x + \sin^2x &= \frac{3}{2} \\
	\iff \cos^2x + \sin(2x) + \sin^2x &= \frac{3}{2} \\
	\iff 2(\cos^2x+ \sin(2x) + \sin^2x) &=  3 \\
	\iff 2\cos^2x + 2\sin(2x) + 2\sin^2x &= 3 \\
	\iff 2\cos^2x - 1 + 1 + 2\sin(2x) + 2\sins^2x &= 3 \\
	\iff \cos(2x) + 1 + 2\sin(2x) + 2\sin^2x &=  3 \\
	\iff \cos(2x) + 2\sin(2x) + 2\sin^2x &= 2 \\
	\iff 2(\frac{1}{2}\cos(2x) + \sin(2x) + \sin^2x) &=  2 \\
	\frac{1}{2}\cos(2x) + \sin(2x) + \sin^2x &=  1 \\
						 &\text{ma vie est une blague lol}
.\end{align*}

\section{Calculus@54 79} %difficulty=1

\begin{align*}
	\tan x &= \frac{\sin x}{\cos x} \\
	\cos (x+y) &= \cos x \cos y - \sin x \sin y \\
	\sin(x+y) &= \sin x \cos y + \sin y \cos x \\
	\tan(x+y) &= \frac{\sin x \cos y + \sin y \cos x}{\cos x \cos y - \sin x \sin y} \\
		  &= \frac{\cos y (\sin x + \frac{\sin y \cos x}{\cos y}}{\cos y (\cos x - \frac{\sin x \sin y}{\cos y}} \\
		  &= \frac{\sin x + \tan y \cos x}{\cos x + \sin x \tan y} \\
		  &= \frac{\cos x \left( \frac{\sin x}{\cos x} + \frac{\tan y \cos x}{\cos x} \right) }{\cos x \left( 1 + \frac{\sin x \tan y}{\cos x} \right) } \\
		  &= \frac{\tan x + \tan y}{1 + \tan x \tan y} \\
.\end{align*}

\section{Calculus@54 80} %difficulty=1
\begin{align*}
	\tan (2x) &= \tan ( x+x) \\
	&= \frac{\tan x + \tan x}{1+\tan x \tan x} \\
	&= \frac{2\tan x}{1+\tan^2x} \\
.\end{align*}

On en déduis donc la valeur de $\tan\frac{\pi}{8}$:

\begin{align}
	\tan \frac{\pi}{4} &= \frac{2\tan\frac{\pi}{8}}{1+\tan^2\frac{\pi}{8}} \\
	\iff 1 &= \frac{2\tan\frac{\pi}{8}}{1+\tan^2 \frac{\pi}{8}} \\
	\iff 1+\tan^2 \frac{\pi}{8} &= 2\tan\frac{\pi}{8} \\
	\iff \tan^2 \frac{\pi}{8} - 2\tan\frac{\pi}{8} +1 &= 0
.\end{align}

On remarque que $(4)$ est de la forme $x^2+2x+1=0$: On cherche donc la/les racines du polynôme du second degré $x^2+2x+1$.

\begin{align*}
	\Delta &= 4 - 4 \\
	       &= 0 \\
	       \implies x_0 &= \frac{-2}{2} \\
	       &= -1 \\
.\end{align*}

On a donc:

\begin{align*}
	\tan^2\frac{\pi}{8} - 2\tan\frac{\pi}{8}+1 &= 0 \\
	\iff \tan\frac{\pi}{8} &= -1 \\
.\end{align*}

Sauf que c'est faux. Et je sais pas pourquoi.

\section{Calculus@54 81} %difficulty=1
On sait que:
\begin{align*}
	\frac{1}{\cos^2x} &= 1+\tan^2x \\
.\end{align*}

\begin{align*}
	\frac{1}{\cos^2x} &= 1+\tan^2x \\
	\iff \cos^2x &= \frac{1}{1+\tan^2x} \\
	\iff \cos x &= \frac{\frac{1}{1+\tan^2x}}{\cos x} \\
	&= \frac{1}{1+\tan^2x\cos x} \\
	&= \frac{1}{1+\tan^2x\cos x} \frac{1-\tan^2x}{1-\tan^2x}\\
	&= \frac{1-\tan^2x}{(1+\tan^2x\cos x)(1-\tan^2x)} \\
	&= \frac{1-\tan^2x}{1+\tan^2x\cos x -\tan^2x -\tan^2x(\tan^2x\cos x)} \\
	&= \frac{1-\tan^2x}{1+\tan^2x \cos x -\tan^2x -\tan^{4}x -\tan^2x \cosx} \\
	&\text{fml}
.\end{align*}

\begin{align*}
	\sin x &= \frac{2t}{1+t^2} \\
	       &\text{Nan mais ça sert plus à rien de continuer là jvais juste apprendre des formules par coeur m\^eme la correction je la comprends pas depuis genre ~10 exos}
.\end{align*}


\end{document}
