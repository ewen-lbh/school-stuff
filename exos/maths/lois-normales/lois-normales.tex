\documentclass{article}
\usepackage[utf8]{inputenc}
\usepackage[a4paper, total={7.5in, 10.5in}]{geometry}
\usepackage{amsmath, amssymb}
\usepackage{chemfig}
\usepackage{chemformula}
\usepackage{float}
\usepackage{dsfont}
\usepackage{cancel}
\usepackage{pgfplots}
\usepackage{dsfont}
\usepackage{multirow}
\usepackage{caption}
\usepackage{calc}  
\usepackage{enumitem}  
\usepackage{graphicx}
\usepackage{scalerel,stackengine}
\stackMath
\newcommand\reallywidehat[1]{%
\savestack{\tmpbox}{\stretchto{%
  \scaleto{%
    \scalerel*[\widthof{\ensuremath{#1}}]{\kern-.6pt\bigwedge\kern-.6pt}%
    {\rule[-\textheight/2]{1ex}{\textheight}}%WIDTH-LIMITED BIG WEDGE
  }{\textheight}% 
}{0.5ex}}%
\stackon[1pt]{#1}{\tmpbox}%
}
\parskip 1ex
\graphicspath{ {./} }
\usetikzlibrary{shapes.arrows}
\let\ce\ch
\newcommand{\im}{\text{Im}\,}
\newcommand{\re}{\text{Re}\,}
\newcommand{\img}{\text{Img}\,}
\newcommand{\R}{\mathds{R}}
\newcommand{\C}{\mathds{C}}
\newcommand{\N}{\mathds{N}}
\newcommand{\Z}{\mathds{Z}}
\newcommand{\conj}[1]{\overline{#1}}
\newcommand{\Aff}{\text{Aff}}
\newcommand{\oo}{\infty}
\newcommand{\twoRows}[1]{\multirow{2}{*}{#1}}
\newcommand{\threeRows}[1]{\multirow{3}{*}{#1}}
\newcommand{\twoCols}[1]{\multicolumn{2}{c|}{#1}}
\newcommand{\threeCols}[1]{\multicolumn{3}{|c|}{#1}}
\newcommand{\twoColsNB}[1]{\multicolumn{2}{c}{#1}}
\newcommand{\goesto}[2]{\xrightarrow[#1\:\to\:#2]{}}
\newcommand{\liminfty}{\lim_{x\to+\oo}}
\newcommand{\limminfty}{\lim_{x\to-\oo}}
\newcommand{\limzero}{\lim_{x\to0}}
\newcommand{ \const}{\text{cste}}
\newcommand{\et}{\:\text{et}\:}
\newcommand{\ou}{\:\text{ou}\:}
\newcommand{\placeholder}{\diamond}
\newcommand{\mediateur}{\:\text{med}\:}
\newcommand{\milieu}{\:\text{mil}\:}
\newcommand{\vect}[1]{\overrightarrow{#1}}
\newenvironment{descriptiona}{\begin{description}[leftmargin=!,labelwidth=\widthof{\bfseries The longest label}]}{\end{description}}
\renewcommand{\arraystretch}{1.4}
\newcommand{\point}[2]{(#1;\;#2)}
\newcommand{\spacepoint}[3]{\begin{pmatrix}#1\\#2\\#3\end{pmatrix}}
\newcommand{\equa}[1]{\begin{equation*}\begin{split}#1\end{split}\end{equation*}}

\title{Exercices: Lois normales}
\author{Ewen Le Bihan}
\date{2020-05-15}

\begin{document}
\maketitle

\abstract{}
Chaque section est un exercice, le nom de la section représente a le format suivant:
\texttt{@<page> <exercice>}, avec \texttt{<exercice>} le numéro de l'exercice et \texttt{<page>} le numéro de la page.

Par défaut, les exercices du livre le plus courant sont assumés, mais l'on peut préciser le livre avec la syntaxe suivante:

\texttt{<livre>@<page> <exercice>}.
Exemple:

\texttt{Sésamath@218 24} représente l'exercice numéro \emph{24} du livre \emph{Sésamath} à la page \emph{218}

\section*{@339 76}
On cherche $u_{0.2}$ tel que $P(X \in [-u_{0.2}, u_{0.2}]) = 0.8$
\equa{
	P(X \in [-u_{0.2}, u_{0.2}]) &= 0.8 \\
	\iff 2\phi(u_{0.2})-1 &= 0.8 \\
	\iff \phi(u_{0.2}) &= \frac{1.8}{2} \\
			   &= 0.9 \\
	\iff u_{0.2} &\approx 1.664
}



\section*{@339 73}
\subsection*{a}
\equa{
	P(X \in [-6, 0]) &= \frac{1}{2}
}
\subsection*{b}
\equa{
	P(X > 12) &= 0
}
\subsection*{c}
\equa{
	X &\sim \mathfrac{N}(0,1) \\
	\implies \forall n \in \R P(X=n) &= 0 \\
	\implies P(X=e) &= 0 
}
\subsection*{d}
\equa{ P(X < e) &\approx 0.999 }
\subsection*{e}
\equa{
	P(X \in [-\sqrt{2}, \sqrt{3}]) &\approx 0.879
}
\subsection*{f}
\equa{
	P(X \leq 1.5 | X > 0) &= \frac{P(X > 0 \cap X \leq 1.5)}{P(X > 0)} \\
			      &= \frac{P(X \in [0, 1.5])}{P(X > 0)} \\
			      &\approx \frac{0.433}{0.5} \\
			      &\approx 0.866 
}



\section*{@339 74}
\subsection*{1}
\equa{
	P(X < a) &= \frac{1}{3} \\
	\iff  a  &= -0.431
}
\subsection*{2}
\equa{
	P(X > b) &= \frac{3}{7} \\
	\iff  b  &= 0.143
}
\subsection*{3}
\equa{
	P(X \in [-c, c]) &= \frac{9}{11} \\
	\iff 2\phi(c) - 1 &= \frac{9}{11} \\
	\iff \phi(c) &= \frac{10}{11} \\
	\iff c &= 1.691
}
\end{document}
