\documentclass{article}
\usepackage[utf8]{inputenc}
\usepackage[a4paper, total={6.5in, 9.5in}]{geometry}
\usepackage{amsmath, amssymb}
\usepackage{chemfig}
\usepackage{chemformula}
\usepackage{float}
\usepackage{dsfont}
\usepackage{cancel}
\usepackage{pgfplots}
\usepackage{dsfont}
\usepackage{multirow}
\usepackage{caption}
\usepackage{calc}  
\usepackage{enumitem}  
\usepackage{graphicx}
\usepackage{scalerel,stackengine}
\stackMath
\newcommand\reallywidehat[1]{%
\savestack{\tmpbox}{\stretchto{%
  \scaleto{%
    \scalerel*[\widthof{\ensuremath{#1}}]{\kern-.6pt\bigwedge\kern-.6pt}%
    {\rule[-\textheight/2]{1ex}{\textheight}}%WIDTH-LIMITED BIG WEDGE
  }{\textheight}% 
}{0.5ex}}%
\stackon[1pt]{#1}{\tmpbox}%
}
\parskip 1ex
\graphicspath{ {./} }
\usetikzlibrary{shapes.arrows}
\let\ce\ch
\newcommand{\im}{\text{Im}\,}
\newcommand{\re}{\text{Re}\,}
\newcommand{\img}{\text{Img}\,}
\newcommand{\R}{\mathds{R}}
\newcommand{\C}{\mathds{C}}
\newcommand{\N}{\mathds{N}}
\newcommand{\Z}{\mathds{Z}}
\newcommand{\conj}[1]{\overline{#1}}
\newcommand{\Aff}{\text{Aff}}
\newcommand{\oo}{\infty}
\newcommand{\twoRows}[1]{\multirow{2}{*}{#1}}
\newcommand{\threeRows}[1]{\multirow{3}{*}{#1}}
\newcommand{\twoCols}[1]{\multicolumn{2}{c|}{#1}}
\newcommand{\threeCols}[1]{\multicolumn{3}{|c|}{#1}}
\newcommand{\twoColsNB}[1]{\multicolumn{2}{c}{#1}}
\newcommand{\goesto}[2]{\xrightarrow[#1\:\to\:#2]{}}
\newcommand{\liminfty}{\lim_{x\to+\oo}}
\newcommand{\limminfty}{\lim_{x\to-\oo}}
\newcommand{\limzero}{\lim_{x\to0}}
\newcommand{ \const}{\text{cste}}
\newcommand{\et}{\:\text{et}\:}
\newcommand{\ou}{\:\text{ou}\:}
\newcommand{\placeholder}{\diamond}
\newcommand{\mediateur}{\:\text{med}\:}
\newcommand{\milieu}{\:\text{mil}\:}
\newcommand{\vect}[1]{\overrightarrow{#1}}
\newenvironment{descriptiona}{\begin{description}[leftmargin=!,labelwidth=\widthof{\bfseries The longest label}]}{\end{description}}
\renewcommand{\arraystretch}{1.4}
\newcommand{\point}[2]{(#1;\;#2)}
\newcommand{\spacepoint}[3]{\begin{pmatrix}#1\\#2\\#3\end{pmatrix}}

\title{Exercices: Équations différentielles}
\author{Ewen Le Bihan}
\date{2020-06-05}

\begin{document}
\maketitle

\abstract{}
Exercices provenant de \texttt{./sujet.pdf}

\[
	y' + a(x)y = 0 \iff y = x \mapsto C \exp -A(x)
\] 


\section{Montrer que $y$ est solution de l'équation}
\subsection{}

$y'-2y=0$ avec  $y:x\mapsto e^{2x}$

\begin{align*}
	y' - 2y &= 0 \\
	\iff 2e^{2x} - 2e^{2x} &= 0 \\
\end{align*}

\subsection{}

\section{Résoudre une équation différentielle d'ordre 1 homogène}

\emph{Dans chaque exercice, $C \in \R$} 

\subsection{$y'+\frac{2}{x^2}y$}

\begin{align*}
	y &= x \mapsto C\exp\left( {2\cdot \frac{1}{-1}x^{-1}} \right)  \\
	  &= x \mapsto C \exp\left( -\frac{2}{x} \right)  \\
\end{align*}

\subsection{}

\begin{align*}
	(x^2 + x+1)y' + (2x+1)y &= 0 \\
	\iff y' + \frac{2x+1}{x^2+x+1}y &= 0 \\
	\iff y &= x \mapsto C \exp(-\ln x^2+x+1) \\
\end{align*}

\subsection{}

\begin{align*}
	y' + 2xe^{-x^2}y &= 0 \\
	\iff y &= x \mapsto C \exp (-(-x^2)) \\
	    &= x \mapsto C \exp x^2 \\
\end{align*}

\subsection{}
\begin{align*}
	(x^2+4x+1)^{5}y' - (x+2)y &= 0 \\
	\iff y' - \frac{x+2}{(x^2+4x+1)^{5}}y &= 0 \\
	\iff y &= x \mapsto C\exp \left( -\left(-\frac{1}{2}\cdot \frac{1}{4(x^2+4x+1)^{4}}\right)  \right)  \\
	\iff y&= x \mapsto C \exp\left( \frac{1}{8(x^2+4x+1)^{4}} \right)  \\
\end{align*}

\subsection{}
\begin{align*}
	\sqrt{x^2+1} y' - xy &= 0 \\
	\iff y' - \frac{x}{\sqrt{x^2+1} } &= 0 \\
	\iff y &= x \mapsto C\exp(-(-\frac{1}{2}\cdot 2\sqrt{x^2+1}))  \\
	       &= x \mapsto C\exp \sqrt{x^2+1} \\
\end{align*}

\subsection{}
\begin{align*}
	y'-\cos(3x+1)y&= 0 \\
	\iff y &= x\mapsto C\exp\left(\frac{1}{3} \cdot \sin(3x+1)\right) \\
\end{align*}

\section{Résoudre une équation différentielle d'ordre 1 non-homogène}
\subsection{}

\begin{align*}
	y'-3y&= 5 \\
	% TODO
\end{align*}

\subsection{}

\begin{align*}
	2y'-4y&= 1 \\
	% TODO
\end{align*}

\subsection{}

\begin{align*}
	10y'&=2y -3 \\
	% TODO
\end{align*}




\end{document}
