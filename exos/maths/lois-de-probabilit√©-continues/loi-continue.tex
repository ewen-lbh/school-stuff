\documentclass{article}
\usepackage[utf8]{inputenc}
\usepackage[a4paper, total={6.5in, 9.5in}]{geometry}
\usepackage{float}
\usepackage{amsmath}
\usepackage{amssymb}
\usepackage{mathtools}
\usepackage{tensor}
\newcommand{\torseur}[7]{
\tensor[_{#1}]{\left\{ \begin{array}{cc}
    #2 & #5 \
    #3 & #6 \
    #4 & #7
\end{array} \right}\}{_{(\vec{x};\vec{y};\vec{z})}}
}
\usepackage{siunitx}
\sisetup{output-decimal-marker={,},group-minimum-digits=4,abbreviations}
\sisetup{inter-unit-product=\ensuremath{{}\cdot{}}}
\newcommand{\deftable}[2]{%
%\hline
\textbf{B.A.M.E}
\begin{table}[h]
    \centering
    \begin{tabular}{llp{130mm}}%
        %& unité/type & Explication \ \hline
        #1
    \end{tabular}
    \label{tab:#2_units}
\end{table}%
}
\newcommand{\deftablevar}[3]{%
    $#1$ & $\si{#2}$ & #3 \
}
\newcommand{\deftableobj}[3]{%
    $#1$ & \textit{#2} & #3 \
}
\newcommand{\bame}[1]{%
%\hline
\begin{table}[h]
    \centering
    \begin{tabular}{llllp{130mm}}%
        Nom & Vecteur & Direction & Sens & Norme \hline
        #1
    \end{tabular}
\end{table}%
}
\newcommand{\vect}[1]{\overrightarrow{#1}}

\title{Lois de probabilités continues: Séance 1}
\author{Ewen Le Bihan}
\date{2020-04-26}

\begin{document}

\maketitle

\section{Exercice 1}

\begin{equation*}
    \begin{split}
        P(C \in [0.5, 0.75]) &= \frac{0.75-0.5}{1-0} \\
                             &= 0.25 \\
        P(C > 0.7) &= P(C \in ]0.7, 1]) \\
                   &= \frac{1-0.7}{1-0} \\
                   &= 0.3
    \end{split}
\end{equation*}

\section{Exercice 2}

\subsection{}

Soit $X$ la variable aléatoire suivant la loi uniforme sur $[9, 19]$.
Si je me suis garé à $10.5$ heures, la probabilité que je reçoive un PV est de:

\begin{equation*}
    \begin{split}
        P(X > 10.5) &= P(X \in ]10.5, 19]) \\
                    &= \frac{19-10.5}{19-9} \\
                    &= \frac{8.5}{10} \\
                    &= 0.85
    \end{split}
\end{equation*}

\subsection{}

\begin{equation*}
    \begin{split}
        P(X > 11.5) &= P(X \in ]11.5, 19]) \\
        &= \frac{19-11.5}{19-9} \\
        &= \frac{7.5}{10} \\
        &= 0.75
    \end{split}
\end{equation*}



\end{document}
