\documentclass{article}
\usepackage[utf8]{inputenc}
\usepackage[a4paper, total={6.5in, 9.5in}]{geometry}
\usepackage{float}
\usepackage{amsmath}
\usepackage{amssymb}
\usepackage{mathtools}
\usepackage{tensor}
\newenvironment{eq}{\begin{equation*}\begin{split}}{\end{split}\end{equation*}}
\newcommand{\torseur}[7]{
\tensor[_{#1}]{\left\{ \begin{array}{cc}
  #2 & #5 \
  #3 & #6 \
  #4 & #7
\end{array} \right}\}{_{(\vec{x};\vec{y};\vec{z})}}
}
\usepackage{siunitx}
\sisetup{output-decimal-marker={,},group-minimum-digits=4,abbreviations}
\sisetup{inter-unit-product=\ensuremath{{}\cdot{}}}
\newcommand{\deftable}[2]{%
%\hline
\textbf{B.A.M.E}
\begin{table}[h]
  \centering
  \begin{tabular}{llp{130mm}}%
    %& unité/type & Explication \ \hline
    #1
  \end{tabular}
  \label{tab:#2_units}
\end{table}%
}
\newcommand{\deftablevar}[3]{%
  $#1$ & $\si{#2}$ & #3 \
}
\newcommand{\deftableobj}[3]{%
  $#1$ & \textit{#2} & #3 \
}
\newcommand{\bame}[1]{%
%\hline
\begin{table}[h]
  \centering
  \begin{tabular}{lllll}%
    Nom & Vecteur & Direction & Sens & Norme \hline
    #1
  \end{tabular}
\end{table}%
}
\newcommand{\vect}[1]{\overrightarrow{#1}}

\title{TD: Fauteuil Roulant}
\author{Ewen Le Bihan}
\date{2020-04-03}

\begin{document}

\maketitle

\section{}

\begin{itemize}
  \item Assistance à la force physique 
\end{itemize}

\section{}

\begin{table}[H]
  \centering
  \begin{tabular}{lllll}
    Nom & Vecteur & Direction & Sens & Norme \\\hline
    Poids & $\vec P_S$ & $\vert$ & $\downarrow$ & $65g$ \\
    Réaction de la grosse roue & $\vec R_A$ & ? & ? & ? \\
    Réaction de la petite roue & $\vec R_B$ & $\vert$ & $\uparrow$ & ?
  \end{tabular}
  \caption{BAME}
  \label{tab:bame}
\end{table}

\begin{equation*}
  \begin{split}
    \sum \vec F_\text{ext} &= m\vec a \\
    \iff \vec P_S + \vec R_A + \vec R_B &= m\vec a
  \end{split}
\end{equation*}

Projection selon $\vec x$

\begin{equation*}
  \begin{split}
    0 + A_x + 0 &= m_S \cdot a \\
    \iff A_x &= m_S \cdot a
  \end{split}
\end{equation*}

Projection selon $\vec y$

\begin{equation*}
  \begin{split}
    -m_S \cdot g + A_y + B_y &= m_S \cdot 0 \\
    \iff -m_S \cdot \vec g + A_y + B_y &= 0 \\
    \iff A_y - m_S \cdot \vec g + B_y &= 0
  \end{split}
\end{equation*}

\begin{equation*}
  \begin{split}
    \sum M_G(\vec F_\text{ext}) &= \vec 0 \\
    \iff M_G(\vec P_S) + M_G(\vec R_A) + M_G(\vec R_B) &= \vec 0 \\
  \end{split}
\end{equation*}

Projection selon $\vec z$

\begin{equation*}
  \begin{split}
    0 + M_G(\vec R_A)_z + M_G(\vec R_B)_z &= \vec 0 \\
    \iff y_G \cdot A_x - x_G \cdot A_y - (AB - x_G) \cdot B_y &= 0 \\
    \iff 0 + 
  \end{split}
\end{equation*}

Lorsque la valeur de l'action en B est de $0$, le moment $M_G(\vec R_A)$ n'est plus compensé, et la somme des moments n'est plus nulle: le fauteuil bascule.

\begin{equation*}
  \begin{split}
    y_G \cdot A_x - x_G \cdot A_y - (AB - x_G) \cdot B_y &= 0 \\
    \iff A_x y_G &= x_G A_y + AB + x_G B_y \\
    \iff A_x &= \frac{x_G A_y + AB + x_G B_y}{y_G} \\
    &= \frac{194 A_y + 336 +  194 \cdot 0}{643} \\
    &= \frac{194A_y + 336}{643} \\
    &= \frac{194}{643}A_y + \frac{336}{643}
  \end{split}
\end{equation*}

Or,
\begin{equation*}
  \begin{split}
    A_y - m_S \cdot g + B_y &= 0 \\
    \iff A_y &= m_S \cdot g - B_y \\
    A_y &= 65 \cdot 9.81 - 0\\
    &= \SI{638}{\newton}
  \end{split}
\end{equation*}

Donc:
\begin{equation*}
  \begin{split}
    A_x &= \frac{194}{643}638 + \frac{336}{643} \\
    &= \SI{193}{\newton}
  \end{split}
\end{equation*}

\section{}

\begin{equation*}
  \begin{split}
    \sum M_O(F_\text{ext}) &= \vec 0 \\
    \iff M_O(\vec R_A) + M_O(\vec \vec R_{\text{utilisateur}\to\text{RM}}) &= \vec 0
  \end{split}
\end{equation*}

$A_x R_\text{roue}$ représente la somme des moments, à laquelle on enlève l'effort apporté par l'utilisateur, $F_\text{main} R_\text{main}$.

\section{}

\begin{equation*}
  \begin{split}
    C_\text{moteur} &= A_x R_\text{main} - F_\text{main} R_\text{main} \\
    &= 193 \cdot 320 - 15 \cdot 250 \\
    &= \SI{5.8e4}{\newton\meter}
  \end{split}
\end{equation*}

\section{}

Ces dispositifs sont utiles pour varier la vitesse de déplacement tout en réduisant les chances de basculement.

\end{document}
