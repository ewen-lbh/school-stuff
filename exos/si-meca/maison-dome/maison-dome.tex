\documentclass{article}
\usepackage[utf8]{inputenc}
\usepackage[a4paper, total={6.5in, 9.5in}]{geometry}
\usepackage{float}
\usepackage{amsmath}
\usepackage{amssymb}
\usepackage{mathtools}
\usepackage{tensor}
\usepackage{titlesec}
\newcommand{\torseur}[7]{
\tensor[_{#1}]{\left\{ \begin{array}{cc}
    #2 & #5 \\
    #3 & #6 \\
    #4 & #7
\end{array} \right\}}{_{(\vec{x};\vec{y};\vec{z})}}
}
\usepackage{siunitx}
\sisetup{output-decimal-marker={,},group-minimum-digits=4,abbreviations}
\sisetup{inter-unit-product=\ensuremath{{}\cdot{}}}
\newcommand{\deftable}[2]{%
%\hline
\textbf{B.A.M.E}
\begin{table}[h]
    \centering
    \begin{tabular}{llp{130mm}}%
        %& unité/type & Explication \\ \hline
        #1
    \end{tabular}
    \label{tab:#2_units}
\end{table}%
}
\newcommand{\deftablevar}[3]{%
    $#1$ & $\si{#2}$ & #3 \\
}
\newcommand{\deftableobj}[3]{%
    $#1$ & \textit{#2} & #3 \\
}
\newcommand{\bame}[1]{%
%\hline
\begin{table}[h]
    \centering
    \begin{tabular}{llllp{130mm}}%
        Nom & Vecteur & Direction & Sens & Norme \\hline
        #1
    \end{tabular}
\end{table}%
}
\newcommand{\vect}[1]{\overrightarrow{#1}}

\titleformat{\section}{}{Q\thesection}{0.5ex}{}

\title{Sciences de l'ingénieur--TD 1: Maison dôme}
\author{Ewen Le Bihan}
\date{2020-03-26}

\begin{document}

\maketitle

\section{}

\begin{equation*}
    \begin{split}
        \sum \vec M_\text{ext}(G) = J_G \cdot \vec \alpha
    \end{split}
\end{equation*}

\section{}
\textbf{B.A.M.E.}

\begin{equation*}
    \begin{split}
        \{T_{2\to1}\} &= \torseur{G}{0}{0}{0}{0}{0}{8000}\\
        \{T_{0\to1}\} &= \torseur{G}{X_{0\to1}}{Y_{0\to1}}{0}{0}{0}{N_{0\to1}}
    \end{split}
\end{equation*}

\textbf{Calcul de l'accélération $\alpha$}

Nous sommes à vitesse uniforme, donc $\alpha = 0$

D'après Q1:

\begin{equation*}
    \begin{split}
        \sum \vect{M_\text{ext}(G)} &= J_G \cdot \vec \alpha \\
        \iff \vect{M_G(0\to1)} + \vect{M_G(2\to1)} &= J_G \cdot \vec 0 \\
        \iff \vect{M_G(0\to1)} &= -\vect{M_G(2\to1)}
    \end{split}
\end{equation*}

Projection sur $\vec z$:

\begin{equation*}
    \begin{split}
        {M_G(0\to1)}_z &= -{M_G(2\to1)}_z \\
                       &= \SI{-8E3}{\newton\meter}
    \end{split}
\end{equation*}

Or $N_{0\to1} = {M_G(0\to1)}_z$, donc:

$$M_G(0\to1) = \torseur{G}{X_{0\to1}}{Y_{0\to1}}{0}{0}{0}{-8000}$$

\section{}

\textbf{Calcul de l'accélération}

Dans cette phase, par lecture graphique:

$$\alpha = \frac{-30\times10^{-4}}{8} = \SI{-6e-4}{\radian\per\second\squared}$$

D'après Q1:

\begin{equation*}
    \begin{split}
        \sum \vect{M_\text{ext}(G)} &= J_G \cdot \vec \alpha \\
    \end{split}
\end{equation*}

Or ici $\vect{M_G(2\to1)} = \vec 0$, donc:

\begin{equation*}
    \begin{split}
        \sum \vect{M_\text{ext}(G)} &= J_G \cdot \vec \alpha \\
        \iff \vect{M_G(0\to1)} &= J_G\cdot \vec\alpha \\
        \iff J_G &= \frac{\vect{M_G(0\to1)}}{\vec\alpha} \\
    \end{split}
\end{equation*}

Projection sur $\vec z$:

\begin{equation*}
    \begin{split}
        J_G &= \frac{M_G(0\to1)}{\alpha} \\
        &= \frac{-8000}{-6\times10^{-4}} \\
        &= \SI{1.3e7}{\kilo\gram\meter\squared}
    \end{split}
\end{equation*}


\end{document}
