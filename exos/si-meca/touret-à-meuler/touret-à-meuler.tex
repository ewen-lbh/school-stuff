\documentclass{article}
\usepackage[utf8]{inputenc}
\usepackage[a4paper, total={6.5in, 9.5in}]{geometry}
\usepackage{float}
\usepackage{amsmath}
\usepackage{amssymb}
\usepackage{mathtools}
\usepackage{tensor}
\newcommand{\torseur}[7]{
\tensor[_{#1}]{\left\{ \begin{array}{cc}
  #2 & #5 \\
  #3 & #6 \\
  #4 & #7
\end{array} \right\}}{_{(\vec{x};\vec{y};\vec{z})}}
}
\usepackage{siunitx}
\sisetup{output-decimal-marker={,},group-minimum-digits=4,abbreviations}
\sisetup{inter-unit-product=\ensuremath{{}\cdot{}}}
\newcommand{\deftable}[2]{%
%\hline
\textbf{B.A.M.E}
\begin{table}[h]
  \centering
  \begin{tabular}{llp{130mm}}%
    %& unité/type & Explication \ \hline
    #1
  \end{tabular}
  \label{tab:#2_units}
\end{table}%
}
\newcommand{\deftablevar}[3]{%
  $#1$ & $\si{#2}$ & #3 \
}
\newcommand{\deftableobj}[3]{%
  $#1$ & \textit{#2} & #3 \
}
\newcommand{\bame}[1]{%
%\hline
\begin{table}[h]
  \centering
  \begin{tabular}{llllp{130mm}}%
    Nom & Vecteur & Direction & Sens & Norme \hline
    #1
  \end{tabular}
\end{table}%
}
\newcommand{\vect}[1]{\overrightarrow{#1}}

\title{SI Mécanique--TD 2: Touret à meuler}
\author{Ewen Le Bihan}
\date{2020-03-27}

\begin{document}

\maketitle

\section{Phase de démarrage}
\subsection{}

Calcul de la vitesse angulaire $\omega_1$ en $\si{\radian\per\second}$:

\begin{equation*}
  \begin{split}
    \omega_1 &= N \cdot \frac{2\pi}{60}\\
           &= 3000 \cdot \frac{2\pi}{60} \\
           &= \SI{314.16}{\radian\per\second}
  \end{split}
\end{equation*}

Le moteur atteint cette vitesse (en partant de $\omega_0 = 0$) après $\Delta T_1 = \SI{1.5}{\second}$, donc

\begin{equation*}
  \begin{split}
    \alpha_1 &= \frac{\omega_1}{\Delta T_1}\\
             &= \frac{314.16}{1.5} \\
             &= \SI{209.44}{\radian\per\second\squared}
  \end{split}
\end{equation*}

\subsection{}
\emph{On déduit que l'échelle du schéma est en $\si{\milli\meter}$}

\subsubsection{Détermination de la masse de la broche}

Soit $V$ le volume, $r$ le rayon et $h$ la hauteur de la broche.
\begin{equation*}
  \begin{split}
    m &= V\rho \\
      &= 2\pi r h \rho_2 \\
      &= 2\pi (\frac{46}{2} \cdot 700)\cdot10^{-3} \cdot 7800 \\
      &= \SI{789.04}{\kilo\gram}
  \end{split}
\end{equation*}


\subsubsection{Calcul de $I_{G,X}$}

\begin{equation*}
  \begin{split}
    I_{G,X} &= \frac{1}{2}m r^2 \\
            &= \frac{1}{2}789.04 \cdot (700\cdot10^{-3})^2 \\
            &= \SI{193.31}{\kilo\gram\meter\squared}
  \end{split}
\end{equation*}

\subsection{}

D'après le principe fondamental de la dynamique:

\begin{equation*}
  \begin{split}
    \vec M_{\text{ext}}(G) &= I_{G,x} \cdot \vec \alpha_1
  \end{split}
\end{equation*}

Le touret effecue une rotation autour de l'axe $\vec x$, donc:

\begin{equation*}
  \begin{split}
    \vec M_{\text{ext}}(G) &= I_{G,x} \cdot \vec \alpha_1 \\
    \iff M_{\text{ext}}(G)_x &= I_{G,x} \cdot \alpha_1 \\
    \iff C_m &= 193.31 \cdot 789.04\quad(1)\\
    &= \SI{152.53}{\newton\meter}
  \end{split}
\end{equation*}

\begin{tabular}{ll}
  (1) & Les frottements sont négligés, donc le seul moment appliqué à la broche est le couple moteur $C_m$.
\end{tabular}

\section{Phase d'arrêt}
\subsection{}

Le moteur atteint une vitesse de $\SI{0}{\radian\per\second}$ en partant de la vitesse nominale $\omega_1 = \SI{314.16}{\radian\per\second}$ après $\Delta T_2 = \SI{40}{\second}$, donc

\begin{equation*}
  \begin{split}
    \alpha_2 &= \frac{-\omega_1}{\Delta T_2}\\
             &= \frac{-314.16}{40} \\
             &= \SI{-7.85}{\radian\per\second\squared}
  \end{split}
\end{equation*}

\subsection{}

D'après le principe fondamental de la dynamique:

\begin{equation*}
  \begin{split}
    \vec M_{\text{ext}}(G) &= I_{G,x} \cdot \vec \alpha_2
  \end{split}
\end{equation*}

Le touret effecue une rotation autour de l'axe $\vec x$, donc:

\begin{equation*}
  \begin{split}
    \vec M_{\text{ext}}(G) &= I_{G,x} \cdot \vec \alpha_2 \\
    \iff M_{\text{ext}}(G)_x &= I_{G,x} \cdot \alpha_2 \\
    \iff C_r &= 193.31 \cdot (-7.85)\\
    &= \SI{-1.52}{\newton\meter}
  \end{split}
\end{equation*}

\subsection{}

Ce couple résistant est dû aux frottements engendrés par le contact physique des meules avec le suppport.

\subsection{}

$|-1.52| \ll |152.53| \implies |\alpha_2| \ll |\alpha_1|$, donc il était bien judicieux de négliger les frottements.

\end{document}
