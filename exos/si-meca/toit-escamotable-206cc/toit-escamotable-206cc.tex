\documentclass{article}
\usepackage[utf8]{inputenc}
\usepackage[a4paper, total={6.5in, 9.5in}]{geometry}
\usepackage{float}
\usepackage{amsmath}
\usepackage{amssymb}
\usepackage{mathtools}
\usepackage{tensor}
\newcommand{\torseur}[7]{
\tensor[_{#1}]{\left\{ \begin{array}{cc}
    #2 & #5 \\
    #3 & #6 \\
    #4 & #7
\end{array} \right\}}{_{(\vec{x};\vec{y};\vec{z})}}
}
\usepackage{siunitx}
\sisetup{output-decimal-marker={,},group-minimum-digits=4,abbreviations}
\sisetup{inter-unit-product=\ensuremath{{}\cdot{}}}
\newcommand{\deftable}[2]{%
%\hline
\textbf{B.A.M.E}
\begin{table}[h]
    \centering
    \begin{tabular}{llp{130mm}}%
        %& unité/type & Explication \ \hline
        #1
    \end{tabular}
    \label{tab:#2_units}
\end{table}%
}
\newcommand{\deftablevar}[3]{%
    $#1$ & $\si{#2}$ & #3 \
}
\newcommand{\deftableobj}[3]{%
    $#1$ & \textit{#2} & #3 \
}
\newcommand{\bame}[1]{%
%\hline
\begin{table}[h]
    \centering
    \begin{tabular}{llllp{130mm}}%
        Nom & Vecteur & Direction & Sens & Norme \hline
        #1
    \end{tabular}
\end{table}%
}
\newcommand{\vect}[1]{\overrightarrow{#1}}

\title{Toit Escamotable 206cc}
\author{Ewen Le Bihan}
\date{2020-05-14}

\begin{document}

\maketitle

\section{}

Nous sommes dans un mouvement plan, donc $L = M = 0$

\begin{equation*}
\begin{split}
    \vect{C_{\text{axe}\to 52}} &= \SI{1800}{\newton} \\
    \vect{A_{10\to 52}} &= d\times F \\
			&= 4 \cdot 1800 \\
			&= \SI{7200}{\newton\meter} \\
    \mathfrac{T} & \torseur{G}{1800}{0}{0}{0}{0}{7200}
\end{split}
\end{equation*}

\section{}

Il y a deux sollicitations existantes:
\begin{itemize}
    \item Traction
    \item Flexion
\end{itemize}

\section{}

Par lecture graphique du document technique, $\sigma_{\max} = \SI{67.34}{\mega\pascal}$.

\begin{equation*}
\begin{split}
    S &= \frac{R_e}{\sigma_{\max}} \\
      &= \frac{500}{67.34} \\
      &= 7.42
\end{split}
\end{equation*}



\end{document}

