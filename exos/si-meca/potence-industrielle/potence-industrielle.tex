\documentclass{article}
\usepackage[utf8]{inputenc}
\usepackage[a4paper, total={6.5in, 9.5in}]{geometry}
\usepackage{float}
\usepackage{amsmath}
\usepackage{amssymb}
\usepackage{mathtools}
\usepackage{tensor}
\newcommand{\torseur}[7]{
\tensor[_{#1}]{\left\{ \begin{array}{cc}
    #2 & #5 \\
    #3 & #6 \\
    #4 & #7
\end{array} \right\}}{_{(\vec{x};\vec{y};\vec{z})}}
}
\usepackage{siunitx}
\sisetup{output-decimal-marker={,},group-minimum-digits=4,abbreviations}
\sisetup{inter-unit-product=\ensuremath{{}\cdot{}}}
\newcommand{\deftable}[2]{%
%\hline
\textbf{B.A.M.E}
\begin{table}[h]
    \centering
    \begin{tabular}{llp{130mm}}%
        %& unité/type & Explication \ \hline
        #1
    \end{tabular}
    \label{tab:#2_units}
\end{table}%
}
\newcommand{\deftablevar}[3]{%
    $#1$ & $\si{#2}$ & #3 \
}
\newcommand{\deftableobj}[3]{%
    $#1$ & \textit{#2} & #3 \
}
\newcommand{\bame}[1]{%
%\hline
\begin{table}[h]
    \centering
    \begin{tabular}{llllp{130mm}}%
        Nom & Vecteur & Direction & Sens & Norme \hline
        #1
    \end{tabular}
\end{table}%
}
\newcommand{\vect}[1]{\overrightarrow{#1}}

\title{Potence industrielle}
\author{Ewen Le Bihan}
\date{2020-05-19}

\begin{document}

\maketitle

\section{}

Soit $H$ un point sur $AC$ avec $AH = x$

\[
	T\{\text{coh}\} = \torseur{H}{0}{24670}{0}{0}{0}{0}
\]

\section{}

\begin{equation*}
	\begin{split}
		\sigma &= \frac{N}{S} \\
		       &= \frac{24670}{(20\cdot2)^2\pi} \\
		       &\approx \SI{79}{\newton\per\milli\meter\squared} \\
		R_{pe} &= \frac{R_e}{s} \\
		       &= \frac{620}{2} \\
		       &\approx \SI{310}{\newton\per\milli\meter\squared} \\
		\sigma &\geq R_{pe} \\
		       &\implies \text{C'est bon}
	\end{split}
\end{equation*}

\section{}

\begin{equation*}
	\begin{split}
		\varepsilon &= \frac{\sigma}{E} \\
			 &= \frac{79}{210 000} \\
			 &= \SI{376E-6}{}
	\end{split}
\end{equation*}

\section{}

\[
	\sigma_{\text{sim}} = \SI{79E6}{\newton\per\meter\squared} \leq \SI{6.024E8}{\newton\per\meter\squared}
\]

\begin{equation*}
	\begin{split}
		R_e = \SI{620E8}{\newton\per\meter\squared} &\leq  \sigma_{\text{sim}} = \SI{1.039E8}{\newton\per\meter\squared}
	\end{split}
\end{equation*}

La résistance est maximale au voisinage du changement de section.

\section{}

\[ 
	\sigma_1 \approx \SI{7.8E7}{\pascal} \approx \SI{7.9E7}{\pascal}
\]

\section{}

\begin{itemize}
	\item Erreurs de mesure
	\item Approximations dans la simulation
	\item Approximations dans la simulation
\end{itemize}

\end{document}
