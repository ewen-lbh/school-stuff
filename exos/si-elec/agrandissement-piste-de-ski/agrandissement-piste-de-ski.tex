\documentclass{article}
\usepackage[utf8]{inputenc}
\usepackage[a4paper, total={6.5in, 9.5in}]{geometry}
\usepackage{float}
\usepackage{amsmath}
\usepackage{amssymb}
\usepackage{mathtools}
\usepackage{tensor}
\newcommand{\torseur}[7]{
\tensor[_{#1}]{\left\{ \begin{array}{cc}
  #2 & #5 \
  #3 & #6 \
  #4 & #7
\end{array} \right\}}{_{(\vec{x};\vec{y};\vec{z})}}
}
\usepackage{siunitx}
\sisetup{output-decimal-marker={,},group-minimum-digits=4,abbreviations}
\sisetup{inter-unit-product=\ensuremath{{}\cdot{}}}
\newcommand{\deftable}[2]{%
%\hline
\textbf{B.A.M.E}
\begin{table}[h]
  \centering
  \begin{tabular}{llp{130mm}}%
    %& unité/type & Explication \ \hline
    #1
  \end{tabular}
  \label{tab:#2_units}
\end{table}%
}
\newcommand{\deftablevar}[3]{%
  $#1$ & $\si{#2}$ & #3 \
}
\newcommand{\deftableobj}[3]{%
  $#1$ & \textit{#2} & #3 \
}
\newcommand{\bame}[1]{%
%\hline
\begin{table}[h]
  \centering
  \begin{tabular}{llllp{130mm}}%
    Nom & Vecteur & Direction & Sens & Norme \hline
    #1
  \end{tabular}
\end{table}%
}
\newcommand{\vect}[1]{\overrightarrow{#1}}

\title{SI: Agrandissement piste de ski}
\author{Ewen Le Bihan}
\date{2020-04-18}

\begin{document}

\maketitle

\section{Q9}

\begin{equation*}
  \begin{split}
    P_{\text{in}} &= \frac{P_\text{out}}{\eta} \\
                  &= \frac{25.2\cdot10^3}{0.9} \\
                  &= \SI{28}{\kilo\watt}
  \end{split}
\end{equation*}

\begin{equation*}
  \begin{split}
    \omega &= \frac{R_1}{V} \\
           &= \frac{1}{2.8} \\
           &= \SI{0.36}{\radian\per\second}
  \end{split}
\end{equation*}

\begin{equation*}
  \begin{split}
    C &= \frac{P_\text{in}}{\omega} \\
      &= \frac{28 000}{0.36} \\
      &= \SI{77777}{\newton\meter}
  \end{split}
\end{equation*}

\section{Q10}

On choisi le moteur 2

\section{Q12}

\begin{equation*}
  \begin{split}
    \omega_s &= R \cdot \omega_e \\
    C_s &= \eta \cdot C_e
  \end{split}
\end{equation*}

\section{Q13}

Le rayon

\section{Q15}

\begin{description}
  \item[Surcouple] $\SI{220}{\newton\meter}$
  \item[Couple] $\SI{180}{\newton\meter}$
\end{description}

\begin{equation*}
  \begin{split}
    C_M &= C_N \cdot 2.9 \\
    &= 196 \cdot 2.9 \\
    &= \SI{568.4}{\newton\meter} \\
    &> \SI{220}{\newton\meter} \\
    &\implies \text{Le surcouple est supporté par le moteur 2}
  \end{split}
\end{equation*}

\begin{equation*}
  \begin{split}
    C_N &= \SI{196}{\newton\meter} \\
    &> \SI{180}{\newton\meter} \\
    &\implies \text{Le couple est supporté par le moteur 2}
  \end{split}
\end{equation*}

Le moteur 2 est un choix acceptable

\section{Q16}

\begin{equation*}
  \begin{split}
    \frac{0.7C_n - 0.2C_n}{0.2C_n} &= \frac{0.7\cdot196 - 0.2\cdot196}{0.2\cdot196} \\
    &= 2.5
  \end{split}
\end{equation*}

\section{Q17}

\begin{description}
  \item[Vit. 2] $\SI{42.5}{\hertz}$
  \item[Vit. 3] $\SI{33}{\hertz}$
\end{description}

\end{document}
