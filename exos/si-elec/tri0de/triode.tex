\documentclass{article}
\usepackage[utf8]{inputenc}
\usepackage[a4paper, total={6.5in, 9.5in}]{geometry}
\usepackage{float}
\usepackage{amsmath}
\usepackage{amssymb}
\usepackage{mathtools}
\usepackage{tensor}
\newcommand{\torseur}[7]{
\tensor[_{#1}]{\left\{ \begin{array}{cc}
    #2 & #5 \
    #3 & #6 \
    #4 & #7
\end{array} \right\}}{_{(\vec{x};\vec{y};\vec{z})}}
}
\usepackage{siunitx}
\sisetup{output-decimal-marker={,},group-minimum-digits=4,abbreviations}
\sisetup{inter-unit-product=\ensuremath{{}\cdot{}}}
\newcommand{\deftable}[2]{%
%\hline
\textbf{B.A.M.E}
\begin{table}[h]
    \centering
    \begin{tabular}{llp{130mm}}%
        %& unité/type & Explication \ \hline
        #1
    \end{tabular}
    \label{tab:#2_units}
\end{table}%
}
\newcommand{\deftablevar}[3]{%
    $#1$ & $\si{#2}$ & #3 \
}
\newcommand{\deftableobj}[3]{%
    $#1$ & \textit{#2} & #3 \
}
\newcommand{\bame}[1]{%
%\hline
\begin{table}[h]
    \centering
    \begin{tabular}{llllp{130mm}}%
        Nom & Vecteur & Direction & Sens & Norme \hline
        #1
    \end{tabular}
\end{table}%
}
\newcommand{\vect}[1]{\overrightarrow{#1}}

\title{Tri'Ode: Sécurité à basse et haute vitesse}
\author{Ewen Le Bihan}
\date{2020-04-30}

\begin{document}

\maketitle

\section{Q8}

\paragraph{Détermination de la taille en octets de la partie "Data" de la trame}

On a neuf champs pour tout les capteurs, chacun transmis sur 4 octets, donc la partie "Data" pèse $9 \times 4 = 36 \;\text{octets}$

\paragraph{Détermination de la taille totale en octets de la trame}

\begin{equation*}
    \begin{split}
        \text{Début de trame} + \text{Identification} + \text{Data} + \text{CS} &= 1 + 3 + 36 + 1 \\
        &= 41\;\text{octets} \\
    \end{split}
\end{equation*}

\paragraph{Détermination de la taille totale en bits du message}

\begin{equation*}
    \begin{split}
        328 (\text{Bit de start} + \text{Data} + \text{Bit de parité} + \text{Bit de stop}) &= 328(1 + 8 + 1 + 1) \\
        &= 41\times11 \\
        &= 450 \;\text{bits}
    \end{split}
\end{equation*}

\paragraph{Calcul de la vitesse}
\begin{equation*}
    \begin{split}
        \frac{450}{19200} &= \SI{24}{\milli\second}
    \end{split}
\end{equation*}

\section{Q9}

$$45 + 24 = 69 \ll 300 \implies \text{Cahier des charges respecté}$$

\end{document}
