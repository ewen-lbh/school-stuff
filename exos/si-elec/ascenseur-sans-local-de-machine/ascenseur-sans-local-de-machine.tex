\documentclass{article}
\usepackage[utf8]{inputenc}
\usepackage[a4paper, total={6.5in, 9.5in}]{geometry}
\usepackage{float}
\usepackage{amsmath}
\usepackage{amssymb}
\usepackage{mathtools}
\usepackage{tensor}
\newcommand{\torseur}[7]{
\tensor[_{#1}]{\left\{ \begin{array}{cc}
    #2 & #5 \
    #3 & #6 \
    #4 & #7
\end{array} \right\}}{_{(\vec{x};\vec{y};\vec{z})}}
}
\usepackage{siunitx}
\sisetup{output-decimal-marker={,},group-minimum-digits=4,abbreviations}
\sisetup{inter-unit-product=\ensuremath{{}\cdot{}}}
\newcommand{\deftable}[2]{%
%\hline
\textbf{B.A.M.E}
\begin{table}[h]
    \centering
    \begin{tabular}{llp{130mm}}%
        %& unité/type & Explication \ \hline
        #1
    \end{tabular}
    \label{tab:#2_units}
\end{table}%
}
\newcommand{\deftablevar}[3]{%
    $#1$ & $\si{#2}$ & #3 \
}
\newcommand{\deftableobj}[3]{%
    $#1$ & \textit{#2} & #3 \
}
\newcommand{\bame}[1]{%
%\hline
\begin{table}[h]
    \centering
    \begin{tabular}{llllp{130mm}}%
        Nom & Vecteur & Direction & Sens & Norme \hline
        #1
    \end{tabular}
\end{table}%
}
\newcommand{\vect}[1]{\overrightarrow{#1}}

\title{Ascenseur sans local de machine}
\author{Ewen Le Bihan}
\date{2020-04-30}

\begin{document}

\maketitle

\section{}

\subsection{}

\begin{itemize}
    \item L'information est transmise à la carte centrale par un bus CAN de terrain
    \item La carte électronique de commande à microprocesseur
    \item transmet des ordres de type logique ou numérique, selon la version de l'ascenseur
\end{itemize}

\subsection{}

\begin{description}
    \item[2] Mécanique de rotation
    \item[3] Mécanique de rotation
\end{description}

\section{}

\begin{itemize}
    \item premier mot de données: $\text{0001}_{16}$, identificateur: $\text{212}_{16}$
    \item premier mot de données: $0000111100110001_2 = \text{F31}_{16}$, identificateur: $182_{16}$
    % \item premier mot de données: $0000111100110001_2 = $, indentificateur: $210_{16} + 2_{16} = 212_{16}$
\end{itemize}

\end{document}
