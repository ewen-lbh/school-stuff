\documentclass{article}
\usepackage[utf8]{inputenc}
\usepackage[a4paper, total={6.5in, 9.5in}]{geometry}
\usepackage{float}
\usepackage{amsmath}
\usepackage{amssymb}
\usepackage{mathtools}
\usepackage{tensor}
\newcommand{\rpm}{\text{tr}\cdot\text{min}^{-1}}
\newcommand{\torseur}[7]{
\tensor[_{#1}]{\left\{ \begin{array}{cc}
  #2 & #5 \
  #3 & #6 \
  #4 & #7
\end{array} \right\}}{_{(\vec{x};\vec{y};\vec{z})}}
}
\usepackage{siunitx}
\sisetup{output-decimal-marker={,},group-minimum-digits=4,abbreviations}
\sisetup{inter-unit-product=\ensuremath{{}\cdot{}}}
\newcommand{\deftable}[2]{%
%\hline
\textbf{B.A.M.E}
\begin{table}[h]
  \centering
  \begin{tabular}{llp{130mm}}%
    %& unité/type & Explication \ \hline
    #1
  \end{tabular}
  \label{tab:#2_units}
\end{table}%
}
\newcommand{\deftablevar}[3]{%
  $#1$ & $\si{#2}$ & #3 \
}
\newcommand{\deftableobj}[3]{%
  $#1$ & \textit{#2} & #3 \
}
\newcommand{\bame}[1]{%
%\hline
\begin{table}[h]
  \centering
  \begin{tabular}{llllp{130mm}}%
    Nom & Vecteur & Direction & Sens & Norme \hline
    #1
  \end{tabular}
\end{table}%
}
\newcommand{\vect}[1]{\overrightarrow{#1}}

\title{TD: Tour transmab (corrigé)}
\author{Ewen Le Bihan}
\date{2020-04-02}

\begin{document}

\maketitle

\section{}

\begin{equation*}
  \begin{split}
    f &= p\cdot n_s \\
    \iff p &= \frac{f}{n_s} \\
    &= 50\frac{60}{1500} \\
    &= 2
  \end{split}
\end{equation*}

\section{}

$$N = 1420\;\rpm \iff \omega = 1420\frac{2\pi}{}$$

\begin{equation*}
  \begin{split}
    g_n &= \frac{n_s - n}{n_s} \\
    &= \frac{1500 - 1420}{1500} \\
    &= 0.0\bar 3\\
    &\text{soit }  5.33\%
  \end{split}
\end{equation*}

\begin{equation*}
  \begin{split}
    P_n &= C_n \omega \\
    \iff C_n &= \frac{P_n}{\omega} \\
    &= 3000 \cdot \frac{60}{2\pi\cdot1420} \\
    &= \SI{20.17}{\newton\meter}
  \end{split}
\end{equation*}

\begin{equation*}
  \begin{split}
    \eta &= \frac{P_\text{mécanique fournie}}{P_\text{électrique absorbée}}\\
    &= \frac{3000}{\sqrt{3}\cdot230\cdot7.1\cdot0.79} \\
    &= 77.1\%
  \end{split}
\end{equation*}

\section{}

\begin{equation*}
  \begin{split}
    N_0 &= \frac{V_c}{R_0} \\
        &= \frac{4}{30\cdot10^{-3}} \\
        &= \SI{133}{\radian\per\second} \\
        &= 1270\;\rpm
  \end{split}
\end{equation*}

\section{}

\begin{equation*}
  \begin{split}
    \Omega &= \frac{4}{0.01} \\
    &= \SI{400}{\radian\per\second} \\
    \implies N &= \frac{\Omega 60}{2\pi} \\
    &= 3820\;\rpm \\
    &< 4000\;\rpm
  \end{split}
\end{equation*}

Donc la vitesse pour $R_1$ n'atteint pas la vitesse maximale du moteur.

\end{document}
