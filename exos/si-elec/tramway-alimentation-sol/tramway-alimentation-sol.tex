\documentclass{article}
\usepackage[utf8]{inputenc}
\usepackage[a4paper, total={6.5in, 9.5in}]{geometry}
\usepackage{float}
\usepackage{amsmath}
\usepackage{amssymb}
\usepackage{mathtools}
\usepackage{tensor}
\newcommand{\torseur}[7]{
\tensor[_{#1}]{\left\{ \begin{array}{cc}
    #2 & #5 \
    #3 & #6 \
    #4 & #7
\end{array} \right}\}{_{(\vec{x};\vec{y};\vec{z})}}
}
\usepackage{siunitx}
\sisetup{output-decimal-marker={,},group-minimum-digits=4,abbreviations}
\sisetup{inter-unit-product=\ensuremath{{}\cdot{}}}
\newcommand{\deftable}[2]{%
%\hline
\textbf{B.A.M.E}
\begin{table}[h]
    \centering
    \begin{tabular}{llp{130mm}}%
        %& unité/type & Explication \ \hline
        #1
    \end{tabular}
    \label{tab:#2_units}
\end{table}%
}
\newcommand{\deftablevar}[3]{%
    $#1$ & $\si{#2}$ & #3 \
}
\newcommand{\deftableobj}[3]{%
    $#1$ & \textit{#2} & #3 \
}
\newcommand{\bame}[1]{%
%\hline
\begin{table}[h]
    \centering
    \begin{tabular}{llllp{130mm}}%
        Nom & Vecteur & Direction & Sens & Norme \hline
        #1
    \end{tabular}
\end{table}%
}
\newcommand{\vect}[1]{\overrightarrow{#1}}

\title{SI: Tramway alimentation par le sol}
\author{Ewen Le Bihan}
\date{2020-04-24}

\begin{document}

\maketitle

\section{}

Mot en binaire: $01101110$\\
Mot en hexadécimal: $6\text{E}$

\section{}

$$\frac{1}{208.3\cdot10{-6}} = 4800 \;\text{bits}\cdot\text{s}{-1}$$

\section{}

Il y a cinq $1_2$ dans le mot transmis, or $\text{mod}(5, 2) \not= 0$, donc le bit de parité doit prendre la valeur de $1_2$.

\paragraph{Protocole complet}
\begin{description}
    \item[Vitesse] $4800 \;\text{bits}\cdot\text{s}{-1}$
    \item[Nombre de bits de données] $8$
    \item[Nombre de bits de start]  $1$
    \item[Nombre de bits de stop]  $2$
    \item[Parité] paire, validée par $1$ bit de parité
\end{description}

L'efficacité est de $\frac{8}{12} = 0.6$ soit $66.67\%$

\section{}

La vitesse de transmission est largement supérieure, l'immunité aux ondes électromagnétique parasites est meilleure.

\end{document}
