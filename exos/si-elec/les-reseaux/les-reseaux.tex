\documentclass{article}
\usepackage[utf8]{inputenc}
\usepackage[a4paper, total={6.5in, 9.5in}]{geometry}
\usepackage{float}
\usepackage{amsmath}
\usepackage{amssymb}
\usepackage{mathtools}
\usepackage{tensor}
\newcommand{\torseur}[7]{
\tensor[_{#1}]{\left\{ \begin{array}{cc}
    #2 & #5 \
    #3 & #6 \
    #4 & #7
\end{array} \right\}}{_{(\vec{x};\vec{y};\vec{z})}}
}
\usepackage{siunitx}
\sisetup{output-decimal-marker={,},group-minimum-digits=4,abbreviations}
\sisetup{inter-unit-product=\ensuremath{{}\cdot{}}}
\newcommand{\deftable}[2]{%
%\hline
\textbf{B.A.M.E}
\begin{table}[h]
    \centering
    \begin{tabular}{llp{130mm}}%
        %& unité/type & Explication \ \hline
        #1
    \end{tabular}
    \label{tab:#2_units}
\end{table}%
}
\newcommand{\deftablevar}[3]{%
    $#1$ & $\si{#2}$ & #3 \
}
\newcommand{\deftableobj}[3]{%
    $#1$ & \textit{#2} & #3 \
}
\newcommand{\bame}[1]{%
%\hline
\begin{table}[h]
    \centering
    \begin{tabular}{llllp{130mm}}%
        Nom & Vecteur & Direction & Sens & Norme \hline
        #1
    \end{tabular}
\end{table}%
}
\newcommand{\vect}[1]{\overrightarrow{#1}}

\title{SI: Les réseaux}
\author{Ewen Le Bihan}
\date{2020-05-12}

\begin{document}

\maketitle

\section{}

Sachant que l'adresse IP la plus petite est 156.18.0.0, et qu'il y avait déjà $242+26=268$ machines connectées, l'adresse IP est 156.18.1.11.

\section{}
\paragraph{}
Convertissons les dimensions en pouces: $\frac{42\cdot10^{-2}\cdot29.6\cdot10^{-2}}{0.0254^2} = \SI{193}{\text{in}\squared}$

\paragraph{}
Calculons ensuite le nombre de points nécéssaires: $600^2\cdot193=69480000\;\text{ppts}$

\paragraph{}
Calculons enfin le nombre de bits nécéssaires, un point étant codé sur un octet, l'imprimmante a besoin de $\SI{69.4}{\mega\text{B}}$ 

\section{}

\begin{equation*}
    \frac{69.4\cdot10^6}{1.3} = \SI{53.4}{\mega\text{B}\per\second}
\end{equation*}

\section{}

On prend la carte ethernet TCP/IP à 23 EUR (soit $14354.30$ Shilling somaliens)

\end{document}
