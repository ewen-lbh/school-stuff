\documentclass{article}
\usepackage[utf8]{inputenc}
\usepackage[a4paper, total={6.5in, 9.5in}]{geometry}
\usepackage{float}
\usepackage{amsmath}
\usepackage{amssymb}
\usepackage{mathtools}
\usepackage{tensor}
\newcommand{\torseur}[7]{
\tensor[_{#1}]{\left\{ \begin{array}{cc}
    #2 & #5 \
    #3 & #6 \
    #4 & #7
\end{array} \right\}}{_{(\vec{x};\vec{y};\vec{z})}}
}
\usepackage{siunitx}
\sisetup{output-decimal-marker={,},group-minimum-digits=4,abbreviations}
\sisetup{inter-unit-product=\ensuremath{{}\cdot{}}}
\newcommand{\deftable}[2]{%
%\hline
\textbf{B.A.M.E}
\begin{table}[h]
    \centering
    \begin{tabular}{llp{130mm}}%
        %& unité/type & Explication \ \hline
        #1
    \end{tabular}
    \label{tab:#2_units}
\end{table}%
}
\newcommand{\deftablevar}[3]{%
    $#1$ & $\si{#2}$ & #3 \
}
\newcommand{\deftableobj}[3]{%
    $#1$ & \textit{#2} & #3 \
}
\newcommand{\bame}[1]{%
%\hline
\begin{table}[h]
    \centering
    \begin{tabular}{llllp{130mm}}%
        Nom & Vecteur & Direction & Sens & Norme \hline
        #1
    \end{tabular}
\end{table}%
}
\newcommand{\vect}[1]{\overrightarrow{#1}}

\title{Machine de tri postal}
\author{Ewen Le Bihan}
\date{2020-05-21}

\begin{document}

\maketitle

\section{}

L'organisation choisie est celle en étoile de David. Ils ont choisi cette organisation car:

\begin{itemize}
	\item L'administration du réseau est facilitée
	\item Les conflits sont évités
\end{itemize}

\section{}

Classe B

\section{}

L'adresse IP est \texttt{172.17.30.4}

\section{}

Ordinateur SI $\to $ Ordinateur GP 

\section{}

Par paquets, de gauche à droite:

\begin{enumerate}
	\item L'information est à destination de l'ordinateur SI
	\item L'information provient de l'ordinateur GP
	\item Le protocole utilisé est TCP/IP
	\item La liaison de supervision est \texttt{ETAT\_PO}
	\item Le message significatif pèse 3 octets
	\item Le message est du type "Exploitation, dépileur en marche"
	\item Le code de défaut est TL (Défaut sur la tête de lecture)
	\item Le défaut est au niveau des cellules de la tête de lecture
	\item (Champ vide)
\end{enumerate}

\section{}

Ethernet est un type de liaison du genre "Réseau local", et autorise une longueur
de câble de plus de $\SI{15}{\meter}$ (l'ordinateur GP est à $\SI{50}{\meter}$, ce 
qui est plus que la limite de la liaison \texttt{RS232}.

\end{document}
