\documentclass{article}
\usepackage[utf8]{inputenc}
\usepackage[a4paper, total={6.5in, 9.5in}]{geometry}
\usepackage{float}
\usepackage{amsmath}
\usepackage{amssymb}
\usepackage{wasysym}
\usepackage{mathtools}
\usepackage{tensor}
\newcommand{\torseur}[7]{
\tensor[_{#1}]{\left\{ \begin{array}{cc}
  #2 & #5 \
  #3 & #6 \
  #4 & #7
\end{array} \right\}}{_{(\vec{x};\vec{y};\vec{z})}}
}
\usepackage{siunitx}
\sisetup{output-decimal-marker={,},group-minimum-digits=4,abbreviations}
\sisetup{inter-unit-product=\ensuremath{{}\cdot{}}}
\newcommand{\deftable}[2]{%
%\hline
\textbf{B.A.M.E}
\begin{table}[h]
  \centering
  \begin{tabular}{llp{130mm}}%
    %& unité/type & Explication \\ \\hline
    #1
  \end{tabular}
  \label{tab:#2_units}
\end{table}%
}
\newcommand{\deftablevar}[3]{%
  $#1$ & $\si{#2}$ & #3 \\
}
\newcommand{\deftableobj}[3]{%
  $#1$ & \textit{#2} & #3 \\
}
\newcommand{\rpm}{\text{tr}\per\minute}
\newcommand{\bame}[1]{%
%\hline
\begin{table}[h]
  \centering
  \begin{tabular}{llllp{130mm}}%
    Nom & Vecteur & Direction & Sens & Norme \hline
    #1
  \end{tabular}
\end{table}%
}
\newcommand{\vect}[1]{\overrightarrow{#1}}

\title{SI: Maison dome}
\author{Ewen Le Bihan}
\date{2020-04-19}

\begin{document}

\maketitle

\section{Q1}

Réseau Domestique 230 V – 50 Hz

\section{Q8}

\begin{equation*}
  \begin{split}
    R_2 &= \frac{Z_1}{Z_2} \frac{Z_3}{Z_4} \frac{Z_5}{Z_6} \\
        &= \frac{51}{50} \frac{13}{41} \frac{13}{66} \\
        &= \frac{1}{15.7}
  \end{split}
\end{equation*}

\section{Q9}

\begin{equation*}
  \begin{split}
    R_G &= R_1 R_2 \\
        &= \frac{1}{141.8} \frac{1}{15.7} \\
        &= \frac{1}{2226}
  \end{split}
\end{equation*}

\section{Q10}

\begin{equation*}
  \begin{split}
    N_2 &= N_m R_G \\
        &= 1450 \frac{1}{2226} \\
        &= \SI{65E-2}{\rpm}
  \end{split}
\end{equation*}

\section{Q11}

\begin{equation*}
  \begin{split}
    R_3 &= \frac{\diameter_p}{\diameter_c} \\
        &= \frac{165\cdot10^{-3}}{3.30} \\
        &= \frac{1}{20}
  \end{split}
\end{equation*}

\begin{equation*}
  \begin{split}
    N_3 &= N_2 \cdot R_3 \\
        &= 65\cdot10^{-2} \cdot \frac{1}{20} \\
        &= \SI{32E-3}{\rpm}
  \end{split}
\end{equation*}

\section{Q12}

\begin{equation*}
  \begin{split}
    N_{\text{TH}} &= \frac{0.5}{60 \cdot 12} \\
    &= \frac{0.5}{720} \\
    &= \SI{6.9E-4}{\rpm}
  \end{split}
\end{equation*}

\section{Q13}

\begin{enumerate}
  \item Rajouter un réducteur afin que $N_3$ soit égal à $N_{\text{TH}}$
  \item Alimenter partiellement le moteur sur des intervalles régulières
\end{enumerate}

\section{Q14}

Rapport entre $N_\text{TH}$ et $N_3$

\begin{equation*}
  \begin{split}
    \frac{N_\text{TH}}{N_3} &= \frac{32\cdot10^{-3}}{6.9\cdot10^{-4}} \\
                            &= 46
  \end{split}
\end{equation*}

On veut que la vitesse soit $46$ fois plus lente, or la fréquence actuelle est de $\frac{1}{10}$

\begin{equation*}
  \begin{split}
    f_s &= \frac{1}{10} \frac{N_\text{TH}}{N_3}10 \\
        &= \frac{1}{10} \frac{6.9\cdot10^{-4}}{32\cdot10^{-3}}10 \\
        &= \SI{0.02}{\hertz}
  \end{split}
\end{equation*}


\section{Q15}




\end{document}
