\documentclass{article}
\usepackage[utf8]{inputenc}
\usepackage[a4paper, total={6.5in, 9.5in}]{geometry}
\usepackage[bookmarks,hidelinks]{hyperref}
\usepackage{amsmath, amssymb}
\usepackage{mathtools}
\usepackage{chemformula}
\usepackage{xcolor}
\definecolor{gray07}{gray}{0.7}
\usepackage{float}
\usepackage{enumitem}
\usepackage{cancel}
\usepackage{tikz}
\usetikzlibrary{decorations.pathreplacing}
\usepackage{multirow}
\usepackage{caption}
\usepackage[load-configurations = abbreviations]{siunitx}
\sisetup{inter-unit-product=\ensuremath{{}\cdot{}}, exponent-product=\ensuremath{{}\cdot{}}, group-minimum-digits=3, range-phrase=-, range-units=single}
\usepackage{graphicx}
\graphicspath{ {./} }
\usetikzlibrary{shapes.arrows}
\let\ce\ch
\newcommand{\bit}{\text{bit}}
\newcommand{\equilibrium}[2]{$\ce{#1}\rightleftharpoons\ce{#2}$}
\newcommand{\soom}{\stackrel{10}{=}}
\newcommand{\vect}{\overrightarrow}
\newcommand{\formulaPyramidFactory}[4]{
    \begin{tikzpicture}[scale=1.25]
\draw (0,0) -- (-0.5, -0.75) node  [align=center, fill=white, midway] {\footnotesize \---};
\draw (0,0) -- (0.5, -0.75) node [align=center, fill=white, midway] {\footnotesize \---};
\draw (0.5, -0.75) -- (-0.5, -0.75) node [align=center, fill=white, midway] {\footnotesize #4};

\draw (0,0) node [above, fill=white] {#1};
\draw (-0.5, -0.75) node [left, fill=white] {#2};
\draw (0.5, -0.75) node [right, fill=white] {#3};

%\draw (0, -1) node [below, align=center] {\small (Pyramide de formule)};
\end{tikzpicture}
}
\newcommand{\formulaPyramid}[3]{\formulaPyramidFactory{#1}{#2}{#3}{$\times$}}
\newcommand{\formulaPyramidAdditive}[3]{\formulaPyramidFactory{#1}{#2}{#3}{$+$}}
\newenvironment{definition}{\begin{description}[leftmargin=!,labelwidth=\widthof{\bfseries Lorem ipsum dolor}]}{\end{description}}
\newcommand{\deftable}[2]{%
%\hline
\begin{table}[h]
    \centering
    \begin{tabular}{llp{100mm}}%
        %& unité/type & Explication \\ \hline
        #1
    \end{tabular}
    \label{tab:#2_units}
\end{table}%
}
\newcommand{\deftablevar}[3]{%
    $#1$ & $\si{#2}$ & #3 \\
}
\newcommand{\deftableobj}[3]{%
    $#1$ & \textit{#2} & #3 \\
}

\title{Électrocinétique}
\date{2020-09-02}
\author{Ewen Le Bihan}

\begin{document}

\maketitle

\section{Conduction du courant électrique}
\section{Charge électrique}
Elle est \emph{quantifiée}\footnote{les valeurs sont toujours des multiples d'une constante}. La constante ici est la charge électrique fondamentale $e = \SI{1.6e-19}{\coulomb}$.

\begin{definition}
	\item[proton] $+e$
	\item[électron] $-e$
\end{definition}

\subsection{Courant électrique}
\paragraph{Définition} Mouvement d'ensemble de charges électriques appelés les "porteurs de charge"

%TODO: diagrame sens du courant/sens des électrons

Au 19ème siècle, on ne sait pas identifier les porteurs de charge, le choix arbitraire qui est fait est celui des charges positives (il est faux), et c'est Hall qui s'en rendra compte et identifie les porteurs de charge, mais on garde la convention.

\subsubsection{Dans les métaux}
Les porteurs de charges sont les électrons libres: chaque atome de conducteur (eg. cuivre, or) libère 1 ou 2 électrons qui peuvent se déplacer librement.

\subsubsection{Dans les liquides}
Les porteurs de charge sont les ions:

\begin{definition}
	\item[cations] charge positive
	\item[anions] charge négative
\end{definition}

\begin{description}
	\item[Électrolyte] solution qui conduit du courant
\end{description}

\subsubsection{Dans les gaz}
En condition normales de température et de pression, un gaz n'est pas conducteur. Mais si la température ou la pression augmente, ou si le gaz est soumis à un champ électromagnétique fort, le gaz deviendra conducteur.

Exemples: les néons, les éclairs,\ldots

\subsection{Intensité du courant électrique $I$}
\paragraph{Définition} Égale à la quantité de charge électrique traversant la section d'un conducteur (eg. un câble électrique) par unité de temps: c'est un \emph{débit de charge}.

Avec $dq$\footnote{Notation $d\ldots$ pour les petites quantités} la charge électrique traversant le conducteur pendant une durée $dt$

\begin{align*}
	I: A &= \frac{dq}{dt} \\
.\end{align*}

\emph{On peut noter que l'unité des ampères est équivalente à des coulombs par seconde} 

\subsubsection{Ordres de grandeur d'intensités}

\begin{definition}
	\item[TGV] 500 A
	\item[Plaque de cuisson] 16 A
	\item[Éclair] 10 000 A
	\item[Smartphone] 100 mA
	\item[CPU] ? nA
\end{definition}

\subsubsection{Application numérique}
\emph{Estimons le nombre d'électrons par seconde qui traversent la section d'un fil parcouru par un courant d'intensité $I = \SI{100}{\milli\ampere}$} 

\begin{align*}
	I &= \frac{dq}{dt} \\
	\iff dq &= I\cdot dt \\
	I\cdot dt &= N\cdot |-e| \\
.\end{align*}

\begin{align*}
	N &= \frac{Idt}{e} \\
	&= \frac{100\cdot 10^{-3}\cdot 1}{1.6\cdot 10^{-19}} \\
	&= \SI{6.25e-17}{} \\
.\end{align*}

\section{ARQS}
\begin{definition}
	\item[régime continu/régime permanent] le flot d'électron dans le conducteur se fait sans changement de débit (ou de vitesse)
	\item[vitesse de déplacement d'un électron] quelques $\si{\milli\meter\per\second}$
\end{definition}

%TODO schéma électrique: lampe -> pile -> interrupteur[ouver] -> (loop)

Lorsque l'on ferme l'interrupteur, la lampe s'allume instantanéement

Le courant électrique de la lampe se comporte comme une onde de célérité $c = \SI{3E8}{\meter\per\second}$.

Si le circuit a une longueur $l=\SI{3}{\centi\meter}$, le temps de propagation de l'onde est:

\begin{align*}
	t &= \frac{l}{c} \\
	&= \frac{3}{3\cdot 10^8} \\
	&= \SI{10e-8}{\second} \\
	&= \SI{10}{\nano\second} \\
.\end{align*}

$\SI{10}{\nano\second}$ entre le premier et le dernier électron

Si on travaille en régime \emph{variable} (l'intensité dépend du temps), avec un courant par ex. périodique, si il a une fréquence $f=\SI{1}{\kilo\hertz}$

\begin{align*}
	T&= \frac{1}{f} \\
	&= \SI{10e-3}{\second} \\
	&= \SI{1}{\milli\second} \\
.\end{align*}

On peut donc négliger la durée de propagation.

En revanche, si $f = \SI{2}{\giga\hertz}$:

\begin{align*}
	T &= \frac{1}{2\cdot 10^9} \\
	&= \SI{0.5e-9}{\second} \\
	&= \SI{0.5}{\nano\second} \\
.\end{align*}

\emph{Dans ces conditions, il faudrait tenir compte de la durée de propagation}

On se placera dans des conditions telles que la durée de propagation soit négligeable, c'est-à-dire tel que:

\begin{align*}
	\frac{l}{c} &\ll T \\
	\iff l &\ll cT \\
	\iff l &\ll \underbrace{\lambda}_{\text{longueur d'onde}}
.\end{align*}

Pour un ordinateur: $f = \SI{1}{\giga\hertz}$

\begin{align*}
	l &\ll \frac{c}{f} \\
	&= \frac{3\cdot 10^8}{10^9} \\
	&= \SI{0.3}{\meter} \\
.\end{align*}

En conclusion, dans l' \emph{approximation des régimes quasi-stationnaires} (ARQS)  (ou régimes lentement variables), le temps de propagation est négligeable.

\section{Définitions}
\begin{definition}
	%TODO: graph câble avec A, B, U_AB et U_BA, V_B et V_A
\item[tension électrique] aussi \emph{ddp}\footnote{différence de potentiel} ou différence d'état électrique: $U_{AB}: \text{V} = V_A - V_B$
\end{definition}

\paragraph{Remarque}
\begin{align*}
	U_{BA} = V_B - V_A = -U_{AB}
.\end{align*}

On fixe souvent arbitrairement dans un circuit une référence pour les potentiels électrique appelée la \emph{masse}, de symbole \texttt{(symbole d'une masse)}, qui signifie que, par convention, $V_{\text{masse}} = 0$

Pour mesure une différence de potentiel, on utilise un voltmètre.

\begin{definition}
	\item[dipôle] composant électrique possédant deux pôles/bornes (eg. lampe, pile)
	\item[nœud] intersection d'au moins 3 fils
	\item[branche] portion de circuit entre 2 nœuds
	\item[maille] succession de branches qui constituent une boucle
	\item[réseau] circuit constitué de plusieurs mailles
\end{definition}

\section{Lois de Kirchhoff}
\subsection{Lois des nœuds}
En régime continu ou en régime variable (mais dans l'ARQS), il ne peut y avoir nin accumulation de charge, ni déficit de charge.

%TODO: schéma nod

\begin{align*}
	i_1+i_2+i_3&=i_5+i_4 \\
	\iff i_1 + i_2 + i_3 - i_4 - i_5 &= 0 \\
.\end{align*}

%TODO: example plus simple avec la "rivière"

\paragraph{Forme généralisée}
\begin{align*}
	\sum_{k}^{} \epsilon_k i_k &= 0 \\
	\text{avec}\quad \epsilon_k &= \begin{cases}
		+1 &\text{$i_k$ arrive dans le nœud} \\
		-1 &\text{$i_k$ part du nœud}
	\end{cases} \\
.\end{align*}

\subsection{Loi des mailles}
%TODO: graphe avec ABCD en carré, quatres dipoles et les flèches de tension

\begin{align*}
	U_{AD}+U_{BA}+U_{CB}+U_{DC} &= V_A-V_D+V_B-V_A+V_C-V_B+V_D-V_C \\
	&= 0 \\
.\end{align*}

(voir sur le polycopié)

\paragraph{Forme générale}

\begin{align*}
	\sum_{k}^{} \epsilon_k U_k &= 0 \\
	\epsilon_k &= \begin{cases}
		+1 &\text{$U_k$ est dans le même sens que le sens de parcours} \\
		-1 &\text{sinon}
	\end{cases} \\
.\end{align*}

\subsection{Puissance}
Soit un dipôle

%TODO: graph dipole \mathcal{D} avec i et u

Énergie électrique reçue:

\begin{align*}
	\delta_w &:= u \cdot dq \\
	\text{or}\quad i &= \frac{dq}{dt} \\
	\text{donc}\quad dq &= i\cdot dt \\
	\text{et}\quad \delta_w: J = i\cdot u\cdot dt
.\end{align*}

\begin{align*}
	P &= u\cdot i \\
.\end{align*}

Si $P > 0$, le dipôle est un récepteur. sinon il fournit, et est un générateur.

\subsection{Tension dans un fil}
Soit un fil

%TODO grapg fil A->B with i

Les point $A$ et $B$ sont au même état électrique: $V_A = V_B \implies U_{AB}=0$, \emph{même si un courant circule}.

Soit un interrupteur ouvert

%TODO graph

\begin{align*}
	I=0
.\end{align*}

car l'interrupteur ouvert n'est traversé par aucun courant.

\paragraph{Attention} $U_{AB}$ n'est cependant pas connue:
une prise EDF a une tension non-nulle, mais nulle lorsque disjonctée.

Lorsque l'interrupteur est fermé, il est équivalent à un fil.

\end{document}
