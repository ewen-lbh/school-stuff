\documentclass{article}
\usepackage[utf8]{inputenc}
\usepackage[a4paper, total={6.5in, 9.5in}]{geometry}
\usepackage[bookmarks,hidelinks]{hyperref}
\usepackage{amsmath, amssymb}
\usepackage{mathtools}
\usepackage{chemformula}
\usepackage{xcolor}
\definecolor{gray07}{gray}{0.7}
\usepackage{float}
\usepackage{enumitem}
\usepackage{cancel}
\usepackage{tikz}
\usetikzlibrary{decorations.pathreplacing}
\usepackage{multirow}
\usepackage{caption}
\usepackage[load-configurations = abbreviations]{siunitx}
\sisetup{inter-unit-product=\ensuremath{{}\cdot{}}, exponent-product=\ensuremath{{}\cdot{}}, group-minimum-digits=3, range-phrase=-, range-units=single}
\usepackage{graphicx}
\graphicspath{ {./} }
\usetikzlibrary{shapes.arrows}
\let\ce\ch
\newcommand{\bit}{\text{bit}}
\newcommand{\equilibrium}[2]{$\ce{#1}\rightleftharpoons\ce{#2}$}
\newcommand{\soom}{\stackrel{10}{=}}
\newcommand{\vect}{\overrightarrow}
\newcommand{\formulaPyramidFactory}[4]{
    \begin{tikzpicture}[scale=1.25]
\draw (0,0) -- (-0.5, -0.75) node  [align=center, fill=white, midway] {\footnotesize \---};
\draw (0,0) -- (0.5, -0.75) node [align=center, fill=white, midway] {\footnotesize \---};
\draw (0.5, -0.75) -- (-0.5, -0.75) node [align=center, fill=white, midway] {\footnotesize #4};

\draw (0,0) node [above, fill=white] {#1};
\draw (-0.5, -0.75) node [left, fill=white] {#2};
\draw (0.5, -0.75) node [right, fill=white] {#3};

%\draw (0, -1) node [below, align=center] {\small (Pyramide de formule)};
\end{tikzpicture}
}
\newcommand{\formulaPyramid}[3]{\formulaPyramidFactory{#1}{#2}{#3}{$\times$}}
\newcommand{\formulaPyramidAdditive}[3]{\formulaPyramidFactory{#1}{#2}{#3}{$+$}}
\newenvironment{definition}{\begin{description}[leftmargin=!,labelwidth=\widthof{\bfseries Lorem ipsum dolor}]}{\end{description}}
\newcommand{\deftable}[2]{%
%\hline
\begin{table}[h]
    \centering
    \begin{tabular}{llp{100mm}}%
        %& unité/type & Explication \\ \hline
        #1
    \end{tabular}
    \label{tab:#2_units}
\end{table}%
}
\newcommand{\deftablevar}[3]{%
    $#1$ & $\si{#2}$ & #3 \\
}
\newcommand{\deftableobj}[3]{%
    $#1$ & \textit{#2} & #3 \\
}

\title{Électrocinétique}
\date{2020-09-02}
\author{Ewen Le Bihan}

\begin{document}

\maketitle

\section{Conduction du courant électrique}
\section{Charge électrique}
Elle est \emph{quantifiée}\footnote{les valeurs sont toujours des multiples d'une constante}. La constante ici est la charge électrique fondamentale $e = \SI{1.6e-19}{\coulomb}$.

\begin{definition}
	\item[proton] $+e$
	\item[électron] $-e$
\end{definition}

\subsection{Courant électrique}
\paragraph{Définition} Mouvement d'ensemble de charges électriques appelés les "porteurs de charge"

%TODO: diagrame sens du courant/sens des électrons

Au 19ème siècle, on ne sait pas identifier les porteurs de charge, le choix arbitraire qui est fait est celui des charges positives (il est faux), et c'est Hall qui s'en rendra compte et identifie les porteurs de charge, mais on garde la convention.

\subsubsection{Dans les métaux}
Les porteurs de charges sont les électrons libres: chaque atome de conducteur (eg. cuivre, or) libère 1 ou 2 électrons qui peuvent se déplacer librement.

\subsubsection{Dans les liquides}
Les porteurs de charge sont les ions:

\begin{definition}
	\item[cations] charge positive
	\item[anions] charge négative
\end{definition}

\begin{description}
	\item[Électrolyte] solution qui conduit du courant
\end{description}

\subsubsection{Dans les gaz}
En condition normales de température et de pression, un gaz n'est pas conducteur. Mais si la température ou la pression augmente, ou si le gaz est soumis à un champ électromagnétique fort, le gaz deviendra conducteur.

Exemples: les néons, les éclairs,\ldots

\subsection{Intensité du courant électrique $I$}
\paragraph{Définition} Égale à la quantité de charge électrique traversant la section d'un conducteur (eg. un câble électrique) par unité de temps: c'est un \emph{débit de charge}.

Avec $dq$\footnote{Notation $d\ldots$ pour les petites quantités} la charge électrique traversant le conducteur pendant une durée $dt$

\begin{align*}
	I: A &= \frac{dq}{dt} \\
.\end{align*}

\emph{On peut noter que l'unité des ampères est équivalente à des coulombs par seconde} 

\subsubsection{Ordres de grandeur d'intensités}

\begin{definition}
	\item[TGV] 500 A
	\item[Plaque de cuisson] 16 A
	\item[Éclair] 10 000 A
	\item[Smartphone] 100 mA
	\item[CPU] ? nA
\end{definition}

\subsubsection{Application numérique}
\emph{Estimons le nombre d'électrons par seconde qui traversent la section d'un fil parcouru par un courant d'intensité $I = \SI{100}{\milli\ampere}$} 

\begin{align*}
	I &= \frac{dq}{dt} \\
	\iff dq &= I\cdot dt \\
	I\cdot dt &= N\cdot |-e| \\
.\end{align*}

\begin{align*}
	N &= \frac{Idt}{e} \\
	&= \frac{100\cdot 10^{-3}\cdot 1}{1.6\cdot 10^{-19}} \\
	&= \SI{6.25e-17}{} \\
.\end{align*}

\end{document}
