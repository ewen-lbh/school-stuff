\documentclass{article}
\usepackage{amsmath, amssymb}
\usepackage{graphicx ,caption}
\usepackage{newunicodechar}
\newunicodechar{°}{\degree}
\usepackage[a4paper, total={7in,10in}]{geometry}
\def\arraystretch{2}
\begin{document}

\begin{titlepage}
\begin{center}
\textit{\today}
\vfill
\textbf{\LARGE{Aspect énergétique \& transformation de la matière}\\\Large{Physique --- Chapitre 12}}\\
\vfill
\large{Ewen Le Bihan\\1eS3}
\end{center}
\end{titlepage}

\section{Changements d'états}
\subsection{Les différents changements d'états}
Dû à une agitation des molécules (et donc à une chaleur) plus élevée\\\\
Changements d'états\\
\begin{tabular}{r||c|c|c}
	Départ $\downarrow$ Arrivée $\rightarrow$  &  Solide & Liquide & Gazeux\\
	\hline \hline
	Solide & / & Fusion & Sublimation\\
	\hline
	Liquide & Solidification & / & Vaporisation\\
	\hline
	Gazeux & Condensation & Liquéfaction & /
\end{tabular}
\\\\
Ce qui rend le changement d'état "positif" plus couteux:
\begin{itemize}
	\item Nb de C $\to$ Nb d'interactions de Van der Waals
	\item Présence de liaisons hydrogènes
\end{itemize}
Les alcools demandent plus d'énergie que les alcanes car ils possèdent les liaisons hydrogène
\subsection{Énergies requises}

%\begin{tabular}{l|l|l|l}
%	Nom & Symbole & Unité & Description\\
%	\hline \hline
%	Capacité thermique & $C$ & $J \cdot K^{-1}$ & Énergie nécéssaire pour augmenter la température de 1 $K$\\
%	\hline
%	Énergie thermique massique & $L_{etat}$ & $J \cdot g^{-1}$ & Énergie requise pour faire passer 1 $g$ à $etat$\\
%	\hline
%	Capacité thermique massique & $c_{objet}$ & $J \cdot K^{-1} \cdot g^{-1}$ & Énergie require pour augmenter la température 1 $g$ de $objet$ de 1 $K$
%\end{tabular}
\subsubsection{Énergie massique de changement d'état $L$}
\begin{flalign*}
\underbrace{E}_{J} &= \underbrace{m}_{g}\cdot \underbrace{L_{\acute{e}tat}}_{g\cdot J^{-1}}\\\\
E &= m \underbrace{c}_{\text{\makebox[0pt]{$J\cdot ^\circ C^{-1}\cdot g^{-1}$}}} (\theta_{i} - \theta_{f})\\\\
E &= \underbrace{C}_{J\cdot ^\circ C^{-1}} \cdot \underbrace{(\theta_f - \theta_i)}_{^\circ C}
\end{flalign*}
Une $L$ de changement d'état "inverse" est égal à l'opposé de l'autre:
\begin{flalign*}
\underbrace{L_{sol}}_{\text{\makebox[0pt]{solidification}}} &= - \underbrace{L_{fus}}_{\text{fusion}}
\end{flalign*}
$E$ et $L$ sont du même signe

\subsection{Chaleur}
La chaleur n'augmente pas pendant un changement d'état

\section{Miscibilité dans l'eau}
\subsection{Hydrophilie/phobie}
Un alcool est hydrophile s'il présente ce schéma:
\begin{equation*}
R - OH
\end{equation*}
(avec $R$ une chaine carbonnée)
\\\\
Sinon, il est hydrophobe.
\subsubsection{Endothermie \& Exothermie}
\begin{tabular}{lll}
	Quand on passe à un état plus désordonné, &il faut \textbf{apporter} de l'énergie: &c'est endothermique\\
Quand on passe à un état moins désordonné, &ça \textbf{dégage} de l'énergie: &c'est exothermique
\end{tabular}
\subsection{Miscibilité}
Un alcool \textbf{hydrophile} est miscible dans l'eau. \\ Un alcool \textbf{hydrophobe} ne l'est pas.
\newpage
\section{Distillation fractionnée}
\subsection{But}
Séparer un mélange de liquides en espèces pures.
\subsection{Matériel}
\begin{itemize}
	\item Colonne de Vigreux
	\item Ballon
	\item Chauffe-ballon
	\item[(] Support élévateur
	\item Réfrigérant droit
	\item Erlenmeyer
\end{itemize}
\end{document}