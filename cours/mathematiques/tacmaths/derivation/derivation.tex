\documentclass{article}
\usepackage[utf8]{inputenc}
\usepackage[a4paper, total={6.5in, 9.5in}]{geometry}
\usepackage[bookmarks,hidelinks,unicode]{hyperref}
\usepackage{amsmath, amssymb}
\usepackage{chemformula}
\usepackage{float}
\usepackage{cancel}
\usepackage{multirow}
\usepackage{caption}
\usepackage{calc}  
\usepackage{enumitem}  
\usepackage{graphicx}
\usepackage{scalerel,stackengine}
\stackMath
\newcommand\reallywidehat[1]{%
\savestack{\tmpbox}{\stretchto{%
  \scaleto{%
    \scalerel*[\widthof{\ensuremath{#1}}]{\kern-.6pt\bigwedge\kern-.6pt}%
    {\rule[-\textheight/2]{1ex}{\textheight}}%WIDTH-LIMITED BIG WEDGE
  }{\textheight}% 
}{0.5ex}}%
\stackon[1pt]{#1}{\tmpbox}%
}
\parskip 1ex
\graphicspath{ {./} }
\usetikzlibrary{shapes.arrows}
\let\ce\ch
\newcommand{\im}{\text{Im}\,}
\newcommand{\re}{\text{Re}\,}
\newcommand{\img}{\text{Img}\,}
\newcommand{\R}{\mathbb{R}}
\renewcommand{\C}{\mathbb{C}}
\newcommand{\N}{\mathbb{N}}
\newcommand{\Z}{\mathbb{Z}}
\newcommand{\conj}[1]{\overline{#1}}
\newcommand{\Aff}{\text{Aff}}
\newcommand{\oo}{\infty}
\newcommand{\twoRows}[1]{\multirow{2}{*}{#1}}
\newcommand{\threeRows}[1]{\multirow{3}{*}{#1}}
\newcommand{\twoCols}[1]{\multicolumn{2}{c|}{#1}}
\newcommand{\threeCols}[1]{\multicolumn{3}{|c|}{#1}}
\newcommand{\twoColsNB}[1]{\multicolumn{2}{c}{#1}}
\newcommand{\goesto}[2]{\xrightarrow[#1\:\to\:#2]{}}
\newcommand{\liminfty}{\lim_{x\to+\oo}}
\newcommand{\limminfty}{\lim_{x\to-\oo}}
\newcommand{\limzero}{\lim_{x\to0}}
\newcommand{ \const}{\text{cste}}
\newcommand{\et}{\:\text{et}\:}
\newcommand{\ou}{\:\text{ou}\:}
\newcommand{\placeholder}{\diamond}
\newcommand{\mediateur}{\:\text{med}\:}
\newcommand{\milieu}{\:\text{mil}\:}
\newcommand{\vect}[1]{\overrightarrow{#1}}
\newenvironment{descriptiona}{\begin{description}[leftmargin=!,labelwidth=\widthof{\bfseries The longest label}]}{\end{description}}
\renewcommand{\arraystretch}{1.4}
\newcommand{\point}[2]{(#1;\;#2)}
\newcommand{\spacepoint}[3]{\begin{pmatrix}#1\\#2\\#3\end{pmatrix}}

\title{Tacmaths: Dérivation}
\author{Ewen Le Bihan}
\date{2020-09-01}

\begin{document}
\maketitle

\paragraph{Dans tout le chapitre}
\begin{itemize}
	\item $f:I\to \R$
	\item $I$ est un intervalle non-trivial\footnote{Au moins deux éléments}
	\item $a \in I$
\end{itemize}

\section{Définition}
\subsection{Nombre dérivé en un point}
%TODO: graph

Le nombre dérivé de $f$ en $a$ noté $f'(a)$ est la limite $\lim_{x \to a} \frac{f(x)-f(a)}{x-a}$ si elle existe et est finie

\subsection{Fonction dérivée}
On dit que $f$ est dérivable sur $x$ pour tout $a\in I$, $f'(a)$ existe, auquel cas on a défini une fonction $f'$ et:

\begin{align*}
	f'\begin{cases}
		I&\to \R \\
		a&\mapsto f'(a)
	\end{cases}
.\end{align*}

appellée fonction dérivée de $f$.

\section{Méthode de calcul}
\subsection{Dérivées usuelles}
\subsubsection{}
\begin{align*}
	f \begin{cases}
		\R&\to \R \\
		x&\mapsto x^2
	\end{cases}
.\end{align*} est dérivable sur $\R$ de dérivée

\begin{align*}
	f'\begin{cases}
		\R\to \R\\
		x\mapsto 2x
	\end{cases}
.\end{align*}

\subsubsection{Plus généralement}
\begin{align*}
	f \begin{cases}
		\R \to \R \\
		x\mapsto x^n
	\end{cases}
.\end{align*} est dérivable sur $\R$ de dérivée

\begin{align*}
	f'\begin{cases}
		\R \to \R \\
		x\mapsto nx^{n-1}
	\end{cases}
.\end{align*}

\subsubsection{}
\begin{align*}
	f \begin{cases}
		\R_+^\ast \to \R \\
		x\mapsto \ln x
	\end{cases}q
.\end{align*} est dérivable sur $\R$ de dérivée

\begin{align*}
	f'\begin{cases}
		\R_+^\ast \to \R \\
		x\mapsto \frac{1}{X}
	\end{cases}
.\end{align*}

\subsection{Opérations sur les dérivées}
\paragraph{Théorème}

Soient $u, v : I \to \R$ deux fonctions dérivables sur $I$. Alors:

\subsubsection{}
Soient $\lambda$ et $\mu$ deux réels.

$\lambda + \mu$  est dérivable et $(\lambda u + \mu v)' = \lambda u' + \mu v'$

\subsubsection{}
$u\cdot v$ est dérivable de dérivée $u'v+v'u$

\subsubsection{}
Si $v$ ne s'annule pas, $\sfrac{u}{v}$ est dérivable de dérivée 

\[
	\frac{u'v-v'u}{v^2}
.\] 

\subsection{Dérivation des fonctions}
\paragraph{Notation}
Si $f \in \mathcal{D}(I, \R)$\footnote{$\mathcal{D}^n(I, O)$ est l'ensemble des fonctions $n$-dérivables définies dans $O$ pour $x \in I$}, alors on peut noter:

\[
	\frac{d}{dx}f(x) = f'(x)
.\] 

\paragraph{Example}
\[
	\frac{d}{dx}\left( \cos x \right) = -\sin x
.\] 

\paragraph{Théorème: Dérivation des fonctions composées}
Soient $u: I\to J$ et $v:J\to \R$ dérivable avec $I, J$ des intervalles.

Alors $\begin{cases}
	I\to \R \\
	x\mapsto v(u(x))
\end{cases}$ est dérivable et $ \frac{d}{dx}\left( v(u(x)) \right) = u'(x)\cdot v'(u(x))$.

\paragraph{Examples}

\begin{itemize}
	\item $\left( e^{u} \right)' = u'e^u$
	\item $(\ln u)' = u'\frac{1}{u} = \frac{u'}{u}$
	\item $\left( u^2 \right)' = u'2u = 2u'u$
\end{itemize}

\section{Utilisation}
\subsection{Théorème du signe de la dérivée (TSD)}
On suppose que $f \in \mathcal{D}(I, J)$

\subsection{}
Si on a $f'\ge 0$ (resp. $f'>0$) sur $I$ alors f est croissante (resp. strictement croissante) sur $I$.

\subsection{}
Si on a $f'\le 0$ (resp. $f'<0$) sur $I$ alors f est décroissante (resp. strictement décroissante) sur $I$.

\subsection{}
Si $f'=0$ sur $I$ alors $f$ croissante sur $I$.

\paragraph{Danger}
L'hypothèse "$I$ intervalle" est importante: (\emph{$\R^\ast$ n'est pas une intervalle})

\paragraph{Example}
$\begin{cases}
	\R^\ast \to \R\\
	x\mapsto \frac{1}{x}
\end{cases}$

$f$ n'est pas décroissante \textbf{sur $\R^\ast$}.

\subsection{Signe de la dérivée seconde}
\subsubsection{}
$f$ est dite convexe (resp. concave) sur $I$ lorsque $C_f$ est au dessus (resp. en dessous) de toutes ses tangeantes.
$f$ est convexe (resp. concave) quand, pour tout $x \in D_{f''}$, $f''(x) \le 0$ (resp. $f''(x) \ge 0$).
%TODO: graphs de x^2, exp & ln

\end{document}
